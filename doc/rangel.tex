\section{RANGEL}
\label{sect:rangel}

\noindent This function determines the range of integration over $\omega$
appropriate for the desired accuracy. The formula is:

\begin{equation}
\omega_{max} \ge \alpha_\lambda - {1 \over \lambda} \ln{(a_c)}
\end{equation}

\noindent where $a_c$ is the maximum relative error of the absolute values
of the excitation values:\\

\begin{equation}
{1 \over 4} {
\int_{-\infty}^{+\infty} Q_{\lambda 0}(\epsilon =1,\omega) d\omega
-
\int_{-\omega_{max}}^{+\omega_{max}} Q_{\lambda 0}(\epsilon =1,\omega) d\omega
\over
\int_{-\infty}^{+\infty} Q_{\lambda 0}(\epsilon =1,\omega) d\omega
}
\le
a_c
\end{equation}

\noindent The values of $\alpha_\lambda$ are:\\

\begin{center}
\begin{tabular}{|ll|}
\hline
Multipolarity & $\alpha_\lambda$\\
\hline
E1 & -0.693\\
E2 & 0.203\\
E3 & 0.536\\
E4 & 0.716\\
E5 & 0.829\\
E6 & 0.962\\
M1 & 0.203\\
M2 & 0.536\\
\hline
\end{tabular}
\end{center}

\noindent It also stores in {\em ACC50} the value of $-{\ln\mathrm{(accuracy)}
\over 50}$.\\

\noindent The result is stored in the variable {\em IRA} for each
multipolarity (in common block {\em RNG}).\\
