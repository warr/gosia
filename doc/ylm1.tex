\section{YLM1}
\label{sect:ylm1}

\noindent This function evaluates the odd spherical harmonics. See {\em
YLM} (section \ref{sect:ylm}) for the even spherical harmonics.\\

\begin{eqnarray}
YLM1(2,1)&=&{1 \over \sqrt{4 \pi}} Y_1^0\nonumber\\
YLM1(2,2)&=&{1 \over \sqrt{4 \pi}} Y_1^1\nonumber\\
YLM1(3,1)&=&{1 \over \sqrt{4 \pi}} Y_2^0\nonumber\\
YLM1(3,2)&=&{1 \over \sqrt{4 \pi}} Y_2^1\nonumber\\
YLM1(3,3)&=&{1 \over \sqrt{4 \pi}} Y_2^2\nonumber\\
YLM1(4,1)&=&{1 \over \sqrt{4 \pi}} Y_3^0\nonumber\\
YLM1(4,2)&=&{1 \over \sqrt{4 \pi}} Y_3^1\nonumber\\
YLM1(4,3)&=&{1 \over \sqrt{4 \pi}} Y_3^2\nonumber\\
YLM1(4,4)&=&{1 \over \sqrt{4 \pi}} Y_3^3\nonumber\\
\end{eqnarray}

etc. etc.\\

Note therefore, that YLM1(3,1) = YLM(1,1) etc.\\

Again, we can calculate the values for YLM1 in the same way as before. e.g.
for YLM1(5,3), which is the same as YLM(2,3):\\

\begin{eqnarray}
YLM1(5,3)&=&{1 \over \sqrt{4 \pi}} Y_4^2\nonumber\\
         &=&{3 \sqrt{10} \over 32 \pi} (7 \cos\theta^2 - 1) \sin\theta\nonumber\\
         &=&{1 \over 4 \pi} {3 \sqrt{10} \over 8} (7 \cos\theta^2 - 1)
	 \sin\theta\nonumber\\
\end{eqnarray}

Note that in the code, the ${1 \over 4 \pi}$ term is written as
0.0795774715, but the square root part is evaluated within the Fortran,
unlike the {\em YLM} function, where the whole thing is hard coded.\\

