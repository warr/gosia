\section{SEQ}
\label{sect:seq}

\noindent This function sorts out the levels starting from the top and
working down. i.e. in the order of the cascade. This allows us to take care
of feeding.\\

\noindent It fills out the {\em KSEQ} array in the {\em LEV} common block
for each decay with four values: The initial level, the final level, the
matrix element and the multipolarity.\\

\begin{equation}
\delta_\lambda =
i^{n(\lambda)}
{1 \over (2 \lambda + 1)!! h^{\lambda + 2}}
\sqrt{8 \pi (\lambda + 1) \over \lambda}
\bigg({E_\gamma \over c}\bigg)^{\lambda + 1/2}
{<I_2 \parallel E(M)\lambda \parallel I_1> \over \sqrt{2 I_1 + 1}}
\end{equation}

\noindent where $n(\lambda)$ = $\lambda$ for E$\lambda$ transitions and
$\lambda + 1$ for M$\lambda$ transitions.\\

\noindent In gosia, the constants are hard-coded for each multipolarity:\\

\begin{center}
\begin{tabular}{|ll|}
\hline
Multipolarity & $\delta_\lambda$\\
\hline
E1 & 398.77393 $E_\gamma^{3/2}$ / $\sqrt{2 I_1 + 1}$\\
E2 & 3.5002636 $E_\gamma^{5/2}$ / $\sqrt{2 I_1 + 1}$\\
E3 & 0.023891302 $E_\gamma^{7/2}$ / $\sqrt{2 I_1 + 1}$\\
M1 & 4.1932861 $E_\gamma^{3/2}$ / $\sqrt{2 I_1 + 1}$\\
M2 & 0.036806836 $E_\gamma^{5/2}$ / $\sqrt{2 I_1 + 1}$\\
\hline
\end{tabular}
\end{center}

\noindent Here, the ratio between E$\lambda$ and M$\lambda$ is 95.0981942.
This seems to be approximately 137.05999679 / 1.4399645 (inverse fine
structure constant and $e^2$).\\


\noindent Also, the ratio between E$\lambda$ and E$\lambda+1$ or between
M$\lambda$ and M$\lambda+1$ is 113.92676 = $\sqrt{1/3}$ $\hbar$c, where
$\hbar$c = 197.326931 MeV fm.\\

\noindent In this function, we also evaluate the correction factor for
multiple excitation by calling {\em GF} (see section \ref{sect:gf}), which
we store in {\em GKP} so that it can be used in {\em DECAY} (see section
\ref{sect:decay}). Note that the factor
$|\delta_\lambda|^2 (1 + c(\lambda))$ which is missing from {\em GF} is
included here.\\

\noindent We also calculate the F coefficients by calling {\em F}
(see section \ref{sect:f}) and store that in {\em FP}, so we can use it in
{\em ANGULA} (see section \ref{sect:angula}).\\
