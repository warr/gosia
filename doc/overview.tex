\chapter{Overview}
\label{chapt:overview}

This document describes the internal workings of the Coulomb excitation code
{\em gosia}. Note that it cannot handle effects due to nuclear forces, only
those due to the Coulomb force. Consequently, it is only valid if the
closest distance of approach is large enough that the nuclear forces do not
play a role.\\

The program {\em gosia} uses a semiclassical approach, which works well for
heavy ions, but not for light ions, where Coulomb-nuclear interference
effects become important.\\

The code may logically be divided into two parts:\\

\begin{itemize}
\item Excitation of the excited states in the nucleus due to the
perturbation induced by the Coulomb potential.
\item Deexcitation of the excited states by emission of $\gamma$ rays.
\end{itemize}

\section{Coulomb excitation}

The excitation is handled by first order time-dependent perturbation theory
using the Coulomb potential as perturbing potential. We want to calculate
the rate of excitation of each state of the nucleus.\\

In order to calculate the Coulomb potential at a given time, we need to know
where the projectile ion is relative to the target. This problem is a
classical Kepler problem\footnote{Kepler considered the possible orbits of
one body round another due to the gravitation field. There are different
kinds of orbit possible: elliptical, parabolic and hyperbolic. It is the
hyperbolic one, which interests us}, where the projectile describes a
hyperbolic path. Consequently, we adopt the same parametrization, which is
used for calculating rocket trajectories on a hyperbolic path. Using this
method, we replace the time $t$ with an ``excentricity anomaly'' $\omega$
and integrate over that. See chapter \ref{chapt:kepler}.\\

Since we wish to calculate the angular distribution of the emitted $\gamma$
rays, we will need to know about the orientation of the states. i.e. we need
to calculate the excitation in terms of the magnetic substates.
Consequently, we need to expand the Coulomb potential in the multipole
expansion so that we have it in terms of electric and magnetic multipole
moments..\\

As we know the trajectory of the projectile, we can replace the vector
between the two particles $\vec{r(t)}$ (which changes with time) with a
function of time $t$ since we know how $\vec{r(t)}$ changes with time, and
since we have replaced time with the excentric anomaly, $\omega$ we can
write the Coulomb potential in the multipole expansion as a function of
$\vec{r(\omega)}$ instead.\\

In so doing, it becomes useful to replace the part dependent on $\omega$
with dimensionless collision functions. See chapter \ref{chapt:collision}.

\section{Deexcitation}
