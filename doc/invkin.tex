\section{INVKIN}
\label{sect:invkin}

\noindent This function calculates the angle of the scattered projectile in
the laboratory frame, when the user gives the angle of the recoiling target
nucleus in the laboratory frame. There are two solutions to this problem, so
if Iflag = 1, it selects the larger angle one and if Iflag = 2, the smaller
one. Note that the small angle solution (Iflag = 2) corresponds to very low
energies of the recoiling target nucleus, for which probably either the
particles don't make it out of the target, or they are not detected. So
Iflag = 2 is probably not very useful. Also this routine calculates the
correct value for the kinematic flag IKIN.\\

\noindent This function was written for the OP,INTI option. In this, the
user gives the angles of the meshpoints in terms of the laboratory system of
the detected particle, unlike OP,INTG, where they are given in terms of the
scattered projectile in its laboratory system, which is ill-defined if
recoiling target nuclei are detected and we integrate over a finite energy
loss in the target. In this case, the IKIN flag needs to flip at some angle,
which is quite probably in the middle of the particle detector, so we would
have to divide the particle detector up into two calculations. However, the
exact angle at which IKIN flips depends on the energy, but we are
integrating over energy loss. The OP,INTI option solves this by getting the
user to provide the meshpoints in the laboratory system of the particle
detector and calculating IKIN for each angle and energy in the integration
over energy loss.\\

Let $\theta_{p_{CM}}$ be the angle of the scattered projectile in the centre
of mass frame and $\theta_{p_{Lab}}$ be the equivalent in the laboratory
frame, while $\theta_{t_{CM}}$ and $\theta_{t_{Lab}}$ are the equivalents
for the target recoils.\\

We can write:
\begin{equation}
\tan \theta_{p_{Lab}} = {\sin \theta_{p_{CM}} \over
\tau + \cos \theta_{p_{CM}}}
\label{eq:theta_p_lab}
\end{equation}

and 

\begin{equation}
\tan \theta_{t_{Lab}} =  {\sin \theta_{t_{CM}} \over
\tau_p + \cos \theta_{t_{CM}}}
\label{eq:theta_t_lab}
\end{equation}

where
\begin{equation}
\tau = \tau_p {M_p \over M_t}
\end{equation}

and
\begin{equation}
\tau_p = \sqrt{E_p \over E_{p_{min}}}
\end{equation}

where
\begin{equation}
E_{p_{min}} = E_p - E_x \times a_{red}
\end{equation}

and the reduced mass $a_{red}$ is given by
\begin{equation}
a_{red} = 1 + {M_p \over M_t}
\end{equation}

and $E_p$ is the beam energy, $E_x$ is the energy of the excited state
indicated by the parameter NCM (by default the first excited state), and
$M_p$ and $M_t$ are the projectile and target nuclei masses in AMU,
respectively.\\

Since
\begin{equation}
\theta_{p_{CM}} = \pi - \theta_{t_{CM}}
\end{equation}

we can substitute in Eq. \ref{eq:theta_t_lab} and get

\begin{equation}
\tan \theta_{t_{Lab}} =  {\sin \theta_{p_{CM}} \over
\tau_p - \cos \theta_{p_{CM}}}
\label{eq:theta_t_lab2}
\end{equation}


So the problem is one of inverting Eq. \ref{eq:theta_t_lab2} to calculate
$\theta_{p_{CM}}$ for a given $\theta_{t_{Lab}}$ and then substituting this
value of $\theta_{p_{CM}}$ in Eq. \ref{eq:theta_p_lab} to obtain
$\theta_{p_{Lab}}$, as required.\\

Let $x$ = $\cos \theta_{p_{CM}}$ and $y$ = $\tan \theta_{t_{Lab}}$. Then we
have:

\begin{equation}
y =  {\sqrt{1 - x^2} \over \tau_p - x}
\label{eq:theta_t_lab3}
\end{equation}

or

\begin{equation}
y \tau_p - xy = \sqrt{1 - x^2}
\label{eq:theta_t_lab5}
\end{equation}

Squaring both sides we get a quadratic equation in x:
\begin{equation}
x^2 (1 + y^2) - 2 \tau_p y^2 x + \tau_p^2 y^2 - 1 = 0
\label{eq:theta_t_lab6}
\end{equation}

Solving for $x$ we get:
\begin{equation}
x = {\tau_p y^2 \pm \sqrt{\tau_p^2 y^4 - (1 + y^2) (\tau_p^2 y^2 - 1)}
\over (1 + y^2)}
\label{eq:theta_t_lab7}
\end{equation}

Once we have $x$ = $\cos \theta_{p_{CM}}$, we can substitute into Eq.
\ref{eq:theta_p_lab}.

\begin{equation}
\theta_{p_{Lab}} = \arctan {\sqrt{1 - x^2} \over \tau + x}
\label{eq:theta_p_lab2}
\end{equation}

This still leaves the problem of IKIN. However, we can sort that out too. We
can find the value of $\theta_{p_{CM}}$ for which $\theta_{p_{Lab}}$ has its
maximum by differentiating Eq. \ref{eq:theta_p_lab} and setting it to zero.
Once we have this value, we can substitute in Eq. \ref{eq:theta_t_lab} to
get the corresponding value of $\theta_{t_{Lab}}$, which we can compare with
the value for which we are calculating and from that we can figure out IKIN.\\

So the problem is to evaluate:

\begin{equation}
{d \over d\theta_{p_{CM}}} \left( \arctan \left({\sin \theta_{p_{CM}} \over
\tau + \cos \theta_{p_{CM}}} \right) \right) = 0
\label{eq:dtheta_p_lab}
\end{equation}

In fact, it turns out that we only need to evaluate:

\begin{equation}
{d \over d\theta_{p_{CM}}} \left( {\sin \theta_{p_{CM}} \over
\tau + \cos \theta_{p_{CM}}} \right) = 0
\label{eq:dtheta_p_lab2}
\end{equation}

since the arc tangent only introduces a $1 / (1 + x^2)$ term, which cannot be
zero for real values of $\theta_{p_{CM}}$.\\

let $u$ = $\sin \theta_{p_{CM}}$ and $v$ = $\tau + \cos \theta_{p_{CM}}$.

\begin{equation}
{(\tau + \cos \theta_{p_{CM}})
({d \over \theta_{p_{CM}}}(\sin \theta_{p_{CM}})) -
(\sin \theta_{p_{CM}})
({d \over \theta_{p_{CM}}}(\tau + \cos \theta_{p_{CM}})) 
\over
(\tau + \cos \theta_{p_{CM}})^2} = 0
\label{eq:dtheta_p_lab3}
\end{equation}

or

\begin{equation}
(\tau + \cos \theta_{p_{CM}}) (\cos \theta_{p_{CM}}) -
(\sin \theta_{p_{CM}}) ( - \sin \theta_{p_{CM}})
 = 0
\label{eq:dtheta_p_lab4}
\end{equation}

\begin{equation}
\tau \cos \theta_{p_{CM}} + \cos^2 \theta_{p_{CM}} +
\sin^2 \theta_{p_{CM}} = 0
\label{eq:dtheta_p_lab5}
\end{equation}

\begin{equation}
\cos \theta_{p_{CM}} = -{1 \over \tau}
\label{eq:dtheta_p_lab6}
\end{equation}

Then we just substitute this value of $\theta_{p_{CM}}$ into Eq.
\ref{eq:theta_t_lab} and if the value of $\theta_{t_{Lab}}$ that we have to
evaluate is greater than this value, we set IKIN = 1, otherwise we set IKIN
= 0.

