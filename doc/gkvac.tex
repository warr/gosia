\section{GKVAC}
\label{sect:gkvac}

\noindent This function calculates the nuclear deorientation for a single
level and stores the results of $G_k$ in the {\em VACDP} array of the common
block GKVAC. The only parameter is the level for which we want to
calculate.\\

\noindent It uses {\em GKK} (see section \ref{sect:gkk}) to calculate the
individual time-dependent deorientation coefficients. This in turn calls
{\em XSTATIC} (see section \ref{sect:xstatic}) to evaluate the Gaussian
charge state distribution, and calculates the time-dependent deorientation
coefficients $G_k(t)$, which it stores in {\em GKI}.\\

\noindent If the absolute value of $\Lambda^*$ is less than $10^{-9}$ or if
it is a thin target experiment, we just set all three $G_k$ values for the
level to unity. Otherwise, we call {\em GKK} which stores the values in {\em
GKI} and then we copy them to {\em VACDP}.\\

\noindent Note, that I don't see any reason why {\em GKK} can't store the
results directly in {\em VACDP} as {\em GKI} is not used anywhere else and
the necessary index to the level is already passed to that function.\\

