\section{POMNOZ}
\label{sect:pomnoz}

\noindent The word ``pomnoz'' is Polish for ``multiply'' and this function
is used to perform the expansion of the exponential of the $A$-matrices to
calculate the approximate Coulomb amplitudes.\\

\begin{eqnarray}
\bar{a_p}^{(0)} =& \bar{a}^{(0)}\\
\bar{a_p}^{(n+1)} =& {1 \over n + 1} A \bar{a_p}^{(n)}\nonumber\\
\bar{a}^{(n+1)} =& \bar{a}^{(n)} + \bar{a_p}^{(n+1)}\nonumber\\
\end{eqnarray}

\noindent If the summation doesn't converge, e.g. when the matrix elements
of the $A$-matrix are large, we subdivide the matrix using {\em PODZIEL}
(see section \ref{sect:podziel}).\\

