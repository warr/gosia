\chapter{Dimensionless collision functions}
\label{chapt:collision}

When we use Kepler's method for solving the hyperbolic path described by the
incoming ion, it is normal to substitute ${a \over v_I} \epsilon\sinh\omega
+ \omega$ for time $t$.\\

We also expand the Coulomb potential in the multipole expansion in terms of
the electric and magnetic multipole moments $M(E\lambda,\mu)$ and
$M(M\lambda,\mu)$ and a time dependent function $S_{\lambda\mu}(t)$.
However, after we have transformed variables using $\omega$ instead of $t$
as our main variable, we need to replace $S_{\lambda\mu}(t)$ with a function
of $\omega$ (and $\epsilon$). These functions are the collision functions.

\begin{equation}
Q^E_{\lambda\mu}(\epsilon, \omega) =
a^\lambda
{(2\lambda-1)!! \over (\lambda-1)!}
\big({\pi \over 2\lambda+1}\big)^{1/2}
r(\omega)
S^E_{\lambda\mu}(t(\omega))
\end{equation}

and\\

\begin{equation}
Q^M_{\lambda\mu}(\epsilon, \omega) =
{c \over v}
a^\lambda
{(2\lambda-1)!! \over (\lambda-1)!}
\big({\pi \over 2\lambda+1}\big)^{1/2}
r(\omega)
S^M_{\lambda\mu}(t(\omega))
\end{equation}

The $Q^E_{\lambda\mu}$ function is evaluated in the function \emph{QE} and
the $Q^M_{\lambda\mu}$ function in \emph{QM}. The values calculated by these
functions are stored in the array \emph{ZETA} by the function \emph{SNAKE}.
Note that \emph{ZETA} is also used for other purposes and the Q functions
start in the position LP7.\\

As we do not need to integrate over the full range of $\omega$ in order to
get a reasonable approximation, we only calculate the collision functions
that we need. The function \emph{QRANGE} calculates over what range we need
to evaluate the collision functions. It is called from \emph{SNAKE} and then
\emph{SNAKE} calculates only those collision functions.\\

Table 2.1 of the gosia documentation gives expressions for the collision
functions for different values of $\lambda$ and $\mu$.

\section{The code}

The main entry point is \emph{SNAKE} which calls \emph{QRANGE}, \emph{QE} and
\emph{QM}. However, prior to calling these functions, \emph{FHIP} has to be
called to evaluate the hyperbolic functions $\sinh$ and $\cosh$ for the
values of $\omega$ we want, and the value of $\sqrt{\epsilon^2 - 1}$ which
is needed by both \emph{QE} and \emph{QM} is calculated in \emph{CMLAB}, where
it is stored as the variable \emph{EROOT} in common block \emph{KIN}.


\begin{verbatim}

CMLAB

FHIP

        /- QRANGE
SNAKE  {-- QE
        \- QM
\end{verbatim}
