\section{LAGRAN}
\label{sect:lagran}

\noindent This function interpolates using the Lagrange method. It is
equivalent in function to the {\em SPLNER} function (see section
\ref{sect:splner}) but using a different interpolation method.\\

\noindent This function uses {\em FUNC} for the function to select linear,
logarithmic or square root interpolation and {\em FUNC1} to perform the
inverse (see sections \ref{sect:func} and \ref{sect:func1}).\\

\noindent Let $P(x)$ be a function which approximates a set of points
($x_i$, $y_i$). We can write:\\

\begin{equation}
P(x) = \sum_{j = 1}^n P_j(x)
\end{equation}

\noindent where the coefficients $P_j$ are given by:\\

\begin{equation}
P_j(x) = y_i \prod_{k = 1 , k \ne j}^n {x - x_k \over x_j - x_k}
\end{equation}

\noindent This gives a curve which goes through all the points smoothly,
which can be used for interpolation. Note, however, that it is very poor at
extrapolation and tends to diverge rapidly before the first point or after
the last. In particular, the more data points used to determine the
Lagrangian (i.e. the higher the order of the Lagrangian) the better it is
for interpolation but the worse it is for extrapolation.\\
