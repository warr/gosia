\section{TRINT}
\label{sect:trint}

\noindent This function calculates the sine and cosine integrals (Si and
Ci). Note that it returns -1 times these values. The sine and cosine
integrals are defined as:\\

\begin{equation}
Si(x) = \int_0^x {\sin(t) \over t} dt
\end{equation}

\noindent and:\\

\begin{equation}
Ci(x) = \gamma + \log_e(x) + \int_0^x {\cos(t-1) \over t} dt
\end{equation}

\noindent where $\gamma$ is the Euler gamma = 0.5772156649.\\

\noindent We calculate using three different methods depending on the
magnitude of the argument $x$. If it is less than 1, we use a series
expansion. Above 1, we use a rational approximation and above $10^8$ we
further approximate the rational approximation.\\

\subsection{Small arguments ($x < 1$)}

\noindent For small $x$, we use the expansion:\\

\begin{equation}
\mathrm{Si} = x - {x^3 \over 3! \times 3} + {x^5 \over 5! \times 5} -
{x^7 \over 7! \times 7} + {\ldots}
\end{equation}

\noindent In the code, we use (for small arguments):

\begin{eqnarray*}
{1 \over 3! \times 3} =& {1 \over 18}    =& 0.05555555\\
{1 \over 5! \times 5} =& {1 \over 600}   =& 1.666667E-3\\
{1 \over 7! \times 7} =& {1 \over 35280} =& 2.83446E-5
\end{eqnarray*}

\noindent and:

\begin{equation}
\mathrm{Ci} = \gamma + \log_e(x) - {x^2 \over 2! \times 2} +
{x^4 \over 4! \times 4} - {x^6 \over 6! \times 6} - \ldots
\end{equation}

\noindent Note, however, that in the code we have -Si and $\gamma$ - Ci
rather than Si and Ci.\\

\begin{eqnarray*}
{1 \over 2! \times 2} =& {1 \over 4}      =& 0.25\\
{1 \over 4! \times 4} =& {1 \over 96}     =& 0.0104166\\
{1 \over 6! \times 6} =& {1 \over 4320}   =& 2.31481E-4\\
{1 \over 8! \times 8} =& {1 \over 322560} =& 3.10019E-6
\end{eqnarray*}

\noindent The function {\em POL4} (see section \ref{sect:pol4}) is used to
evaluate these polynomials.\\

\subsection{Larger arguments}

\noindent For larger arguments, the code uses the method given by Abramowitz
and Stegun - Handbook of Mathematical Functions with Formulas, Graphs and
Mathematical Tables, National Bureau of Standards, 9$^{th}$ Ed. page 233.\\

\begin{equation}
f(x) = {1 \over x}
\bigg({x^8 + a_1 x^6 + a_2 x^4 + a_3 x^2 + a_4 \over x^8 + b_1 x^6 + b_2 x^4 + b_3 x^2 + b_4}\bigg)
+ \epsilon(x)
\end{equation}

\noindent where:\\

\begin{center}
\begin{tabular}{|ll|}
\hline
$a_1$ = 38.027264 & $b_1$ = 40.021433\\
$a_2$ = 265.187033 & $b_2$ = 322.624911\\
$a_3$ = 335.677320 & $b_3$ = 570.236280\\
$a_4$ = 38.102495 & $b_4$ = 157.105423\\
\hline
\end{tabular}
\end{center}

\noindent which gives the error $|\epsilon| < 5 \times 10^{-7}$.\\

\begin{equation}
g(x) = {1 \over x^2}
\bigg({x^8 + a_1 x^6 + a_2 x^4 + a_3 x^2 + a_4 \over x^8 + b_1 x^6 + b_2 x^4 + b_3 x^2 + b_4}\bigg)
+ \epsilon(x)
\end{equation}

\noindent where:\\

\begin{center}
\begin{tabular}{|ll|}
\hline
$a_1$ = 42.242855 & $b_1$ = 48.196927\\
$a_2$ = 302.757865 & $b_2$ = 482.485984\\
$a_3$ = 353.018498 & $b_3$ = 1114.978885\\
$a_4$ = 21.821899 & $b_4$ = 449.690326\\
\hline
\end{tabular}
\end{center}

\noindent which gives the error $|\epsilon| < 3 \times 10^{-7}$.\\

Then we use the relations on P. 232 of Abramowitz and Stegun:

\begin{equation}
Si(x) = {\pi \over 2} - f(x) \cos(x) - g(x) \sin(x)
\end{equation}

\noindent and:\\

\begin{equation}
Ci(x) = f(x) \sin(x) - g(x) \cos(x)
\end{equation}

\noindent Note, however, that in the code, we ${\pi \over 2}$ - Si and -Ci
rather than Si and Ci.\\

\noindent {\bf \large Possible bug: this means that the two methods are not
calculating the same thing!!!}\\

\subsection{Very large arguments}

\noindent For very large arguments ($> 10^8$), the $x^8$ term dominates, so
the term inside the brackets for both f(x) and g(x) is unity leaving only:

\begin{equation}
f(x) = {1 \over x}
\end{equation}

\begin{equation}
g(x) = {1 \over x^2}
\end{equation}
