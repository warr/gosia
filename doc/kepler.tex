\chapter{Hyperbolic solution to Kepler problem}
\label{chapt:kepler}

The problem of a charged ion being deflected by the Coulomb field of another
ion is the same as the Kepler problem of hyperbolic orbits. This is a common
problem in rocket science and consequently much of the literature is in
terms of the gravitational force, which Kepler originally considered.\\

We define $\epsilon$, the \emph{orbital excentricity} and $\omega$, the \emph{
eccentric anomaly}. These are defined as:\\

\begin{equation}
\epsilon = {1 \over \sin{\theta_{cm} / 2}}
\end{equation}

where $\theta_{cm}$ is the center-of-mass scattering angle. For the
hyperbolic case $\epsilon$ $>$ 1.\\

And:\\

\begin{equation}
t = {a \over v_I} (\epsilon \sinh \omega + \omega)
\end{equation}

where $t$ is time and $v_I$ is the initial velocity of the incoming ion and
$a$ is the distance of closest approach.\\

So:\\

\begin{equation}
dt = {a \over v_I} (\epsilon \cosh \omega + 1)
\end{equation}

So instead of performing an integral over $t$ from -$\inf$ to +$\inf$, we
integrate instead over $\omega$. It turns out that we don't have to
integrate over a very wide range of omega in order to get a good
approximation, so in the code we truncate.\\

In the Fortran, we make tables of values of $\sinh\omega$ and $\cosh\omega$
in the function \emph{FHIP} in steps of $d\omega$ = 0.03 starting from zero.
Note that we don't need to calculate for negative values of $\omega$ because
we can use the symmetry rules:\\

\begin{eqnarray}
\sinh(-\omega) &=& -\sinh\omega\nonumber\\
\cosh(-\omega) &=& +\cosh\omega\nonumber\\
\end{eqnarray}

These tables are stored in the common block \emph{HIPER} as \emph{SH} for
$\sinh\omega$ and \emph{CH} for $\cosh\omega$. Normally, gosia calculates 365
values, which is the dimension of these arrays, though this can be changed
by changing the parameter LP12. N.B. LP12 must never exceed 365!\\

Since the collision functions are defined in terms of $r(\omega)$, the
function \emph{SNAKE}, which calculates them use the values of $\sinh\omega$
and $\cosh\omega$ calculated in \emph{FHIP}.\\

The adiabatic parameter $\epsilon \sinh \omega + \omega$ is also calculated
in advance and stored in an array. This is done in the function \emph{SETIN}
and the values are stored in the variable \emph{ADB} which is stored in the
common block \emph{ADX}. Again 365 values are normally calculated, though
this can also be changed by changing LP12.\\

The adiabatic parameter appears in the exponent of the expression for the
rate of change of the excitation amplitudes $a_k$ wrt. $\omega$, so this
complex exponent is evaluated in the function \emph{EXPON}.
