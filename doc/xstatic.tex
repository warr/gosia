\section{XSTATIC}
\label{sect:xstatic}

\noindent This function calculates the static part of the deorientation.\\

\noindent Here we use:\\

\begin{equation}
h = {1 \over {1 + (0.012008 \beta Z^{0.45})^{5/3}}}
\end{equation}

\noindent and\\

\begin{equation}
Q_0 = Z h^{3 \over 5}
\end{equation}

\noindent and\\

\begin{equation}
\sigma_Q = \sqrt{(Q_0 (1 - h))}
\end{equation}

\noindent In the code $Q_0$ is called {\em QCEN} and $\sigma_Q$ is {\em
DQ}.\\

\noindent Then we integrate over the range QCEN $\pm$ 3 DQ to
calculate XNOR, which is the normalisation factor needed to ensure that the
sum is unity.\\

\begin{equation}
\mathrm{XNOR} = \sum_{q_i=-3\sigma_Q}^{+3\sigma_Q} e^{-{1\over 2} ({Q_0 - q_i \over
\sigma_Q})^2}
\end{equation}

\noindent which is, of course, an approximation to an integration between
$\pm\infty$.\\

\noindent The value 0.012008 is $v^\prime$/c, where $v^\prime$ is taken from
Nikolaev and Dmitriev, {\em Phys. Lett.} 82A, {\bf 277}, to be 3.6 $\times$
10$^6$ m/s. The 0.45 is the coefficient alpha from the same paper. The power
of 5/3 is 1/k = 1/0.6 from that paper.\\

