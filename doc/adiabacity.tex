\chapter{Adiabacity parameter}
\label{chapt:adiabacity}

We define an adiabacity parameter $\xi_{kn}$ which represents a ratio
between the collision time and the natural nuclear period for a transition.
It is given by:\\

\begin{equation}
\xi_{kn} = Z_1 Z_2 {e^2 \over \hbar c \sqrt{2}}  \sqrt{M_{1,2}}
\big(
{1 \over \sqrt{E_p -s E_k}} - {1 \over \sqrt{E_p -s E_n}}
\big)
\end{equation}

where Z$_1$ and Z$_2$ are the proton numbers of projectile and target, M$_1$
and M$_2$ are the masses, E$_p$ is the bombarding energy and E$_k$ and E$_n$
are the energies of the k$^{th}$ and n$^{th}$ states, respectively and:\\

\begin{equation}
s = 1 + {M_{1,2} \over M_{2,1}}
\end{equation}

Substituting M$_{1,2}$ = m$_u$ A$_{1,2}$, where m$_u$ = 931.494028
MeV/c$^2$, e$^2$ = 1.439976 MeV fm and $\hbar c$ = 197.327 MeV fm. We get:\\

\begin{equation}
{e^2 \over \hbar c}\sqrt{m_u \over 2} = {1.439976 \over 197.327}
\sqrt{931.494028 \over 2} = {1 \over 6.34977}
\end{equation}

So it is this value of 6.34977 which is hard-coded into the Fortran in the
function \emph{LOAD}, which is where the values of $\xi_{kn}$ are evaluated
and stored in a table \emph{XI} in common block \emph{CXI}. Clearly, this only
works if the bombarding energy is in MeV.\\

