\section{TENS}
\label{sect:tens}

\noindent For each level, we take the stastical tensor describing the
polarization of the level (Bten) and rotate it into the new coordinate system
with the z-axis along the beam direction and the x-axis in the plane of the
orbit in such a way, that the x-component of the impact parameter is positive.
The y- axis is defined to form a right-handed system. The results are written
to {\em ZETA}. Note, however, that this has nothing whatsoever to do with
the $\zeta$ that is used in the excitation part of gosia. It just occupies
the same memory at a different time.\\

\noindent The resulting rotated statistical tensors have 28 values for each
level.\\

\begin{equation}
\rho_{k \chi} \rightarrow \sum_{\chi^\prime} \rho_{k \chi^\prime}
D^k_{\chi^\prime \chi} \bigg( {\pi \over 2} , {\pi + \theta \over 2}, \pi
\bigg)
\end{equation}

\noindent and:\\

\begin{equation}
D^k_{\chi^\prime \chi}(\alpha, \beta, \gamma) = 
e^{i \chi \alpha} d^k_{\chi \chi^\prime}(\beta) e^{i \chi^\prime \gamma}
\end{equation}

\noindent and the values of $d^k_{\chi \chi^\prime}$ are evaluated by the
function {\em DJMM} (see setion \ref{sect:djmm}).\\
