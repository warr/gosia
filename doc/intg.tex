\section{INTG}
\label{sect:intg}

\noindent This function performs the main integration.\\


\noindent We use the 4$^{th}$ order Adams-Moulton predictor-corrector method
for solving an ordinary differential equation. We use an adaptive version,
which can change the step size in order to get the desired accuracy. If the
step size for the integral is too small, it calls {\em DOUBLE} (see section
\ref{sect:double}) to double the step size and it it is too large, it calls
{\em HALF} (see section \ref{sect:half}) to halve it.\\

\noindent The predictor is given as:\\

\begin{equation}
y_p(n+1) = y(n) + {h \over 24} (55 f(n) - 59 f(n-1) + 37 f(n-2) - 9 f(n-3))
\end{equation}

\noindent and the corrector is:\\

\begin{equation}
y_c(n+1) = y(n) + {h \over 24} (9 f_p(n+1) + 19 f(n) - 5 f(n-1) + f(n-2))
\end{equation}

\noindent and the error is:\\

\begin{equation}
|E(n+1)| \approx {19 \over 270} (y_p(n+1) - y_c(n+1)))
\end{equation}

\noindent Note that in the code, we use 1/14 = 0.07143 instead of 19/270 =
0.07037.\\

\noindent The values in ARM are part of the Adams-Moulton code. The first
index is the substate, while the second index has a value from 1 to 7 with
the following meanings:\\

\begin{center}
\begin{tabular}{|ll|}
\hline
Index & Meaning\\
\hline
1 & f(n-3)\\
2 & f(n-2)\\
3 & f(n-1)\\
4 & f(n)\\
5 & y(n) initially\\
6 & is not used\\
7 & y$_p$(n+1)\\
\hline
\end{tabular}
\end{center}
