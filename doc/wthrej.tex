\section{WTHREJ}
\label{sect:wthrej}

\noindent This function evaluates a Wigner 3-j symbol. Note that the six
values passed to it are integers containing twice the spin, in order to
allow for half-integer spins. i.e. you have to double the spins before
calling this function.\\

\noindent This function makes use of the table of prime numbers in {\em IP},
the table of sums of factors of primes in {\em KF} and the table of natural
logs of primes in {\em PILOG}, all of which are stored in the common block
{\em FAKUL}. The values in {\em IP} are set up by the function {\em PRIM}
(see section \ref{sect:prim}) and the others are set up in {\em FAKP} (see
section \ref{sect:fakp}). So both of these functions must have been called
prior to calling {\em WTHREJ}.\\

\begin{equation}
\mathrm{WTHREJ} =
\begin{pmatrix}
J1 / 2 & J2 / 2 & J3 / 2\\
M1 / 2 & M2 / 2 & M3 / 2
\end{pmatrix}
\end{equation}

\noindent where J1, J2, J3, M1, M2 and M3 are the values passed to the
function.\\
