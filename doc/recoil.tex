\section{RECOIL}
\label{sect:recoil}

\noindent This function corrects for relativistic effects of recoiling
nucleus. It uses the function {\em ROTATE} (see section \ref{sect:rotate}) to
rotate into the frame of reference of the recoiling nucleus and back.\\

\noindent The angular distribution in the laboratory frame can be represented
in terms of the angular distribution in the frame of the recoiling nucleus
mulitplied by a second order correction in $\beta$:\\

\begin{equation}
{dW^{lab}(\theta, \phi) \over d\Omega_\gamma} = 
\{ (1 - \beta^2) + \beta U + {1 \over 2} \beta^2 V \}
{dW^{RN}(\theta, \phi) \over d\Omega_\gamma}
\end{equation}

\noindent where:\\

\begin{equation}
U = T + 2 \cos(\theta)
\end{equation}

\noindent and\\

\begin{equation}
V = T^3 + 4 \cos(\theta) \cdot T + 6 \cos^2(\theta)
\end{equation}

\noindent and\\

\begin{equation}
T = {1 \over 2} [ \sin(\theta) e^{-i\phi} L_+ - \sin(\theta) e^{i\phi} L_- ]
\end{equation}

\noindent and the raising and lowering operators for spherical harmonics are:\\

\begin{equation}
L_+ Y_{k\chi} = \sqrt{ (k - \chi)(k + \chi + 1)} Y_{k+1 \chi}
\end{equation}

\noindent and\\

\begin{equation}
L_- Y_{k\chi} = \sqrt{ (k + \chi)(k - \chi + 1)} Y_{k-1 \chi}
\end{equation}

\noindent The formalism is that of Lesser {\em et al.} Nucl. Phys. 190 (1972)
597.\\
