%%%%%%%%%%%%%%%%%%%%%%%%%%%%%%%%%%%%%%%%%%%%%%%%%%%%%%%%%%%%%%%%%%%%%%%%%%%%%
\chapter{Spherical harmonics}
\label{chapt:spherical}

The spherical harmonics are calculated using the functions {\em YLM} and
{\em YLM1}. The former calculate the even spherical harmonics, the latter
the odd ones.\\

The function {\em YLM} uses the axial symmetry flag for the experiment,
which it gets from the common block {\em KIN}. If this flag is set, it
only calculates three of the values and then returns. Otherwise, it
calculates all 15 values. Although the code to calculate the three values is
the same in both cases, it is duplicated in the Fortran.\\

\section{YLM}
\begin{eqnarray}
YLM(1,1)&=&{1 \over \sqrt{4 \pi}} Y_2^0\nonumber\\
YLM(1,2)&=&{1 \over \sqrt{4 \pi}} Y_2^1\nonumber\\
YLM(1,3)&=&{1 \over \sqrt{4 \pi}} Y_2^2\nonumber\\
YLM(2,1)&=&{1 \over \sqrt{4 \pi}} Y_4^0\nonumber\\
YLM(2,2)&=&{1 \over \sqrt{4 \pi}} Y_4^1\nonumber\\
YLM(2,3)&=&{1 \over \sqrt{4 \pi}} Y_4^2\nonumber\\
YLM(2,4)&=&{1 \over \sqrt{4 \pi}} Y_4^3\nonumber\\
YLM(2,5)&=&{1 \over \sqrt{4 \pi}} Y_4^4\nonumber\\
YLM(3,1)&=&{1 \over \sqrt{4 \pi}} Y_6^0\nonumber\\
YLM(3,2)&=&{1 \over \sqrt{4 \pi}} Y_6^1\nonumber\\
YLM(3,3)&=&{1 \over \sqrt{4 \pi}} Y_6^2\nonumber\\
YLM(3,4)&=&{1 \over \sqrt{4 \pi}} Y_6^3\nonumber\\
YLM(3,5)&=&{1 \over \sqrt{4 \pi}} Y_6^4\nonumber\\
YLM(3,6)&=&{1 \over \sqrt{4 \pi}} Y_6^5\nonumber\\
YLM(3,7)&=&{1 \over \sqrt{4 \pi}} Y_6^6\nonumber\\
\end{eqnarray}
etc. etc.\\

where Y$_l^m$ are the usual orthognal spherical harmonics:\\

\begin{equation}
Y_l^m(\theta, \phi) = \sqrt{ {2 l + 1 \over 4 \pi} {(l - m)! \over (l + m)!}}
P_l^m(\cos(\theta)) e^{i m \phi}
\end{equation}

and P$_l^m$ are the associated Legendre polynomials.\\

\begin{eqnarray}
P_0^0(\cos \theta)	&=&	1\nonumber\\
P_1^0(\cos \theta)	&=&	\cos \theta\nonumber\\
P_1^1(\cos \theta)	&=&	-\sin \theta\nonumber\\
P_2^0(\cos \theta)	&=&	{1 \over 2} (3 \cos^2 \theta - 1)\nonumber\\
P_2^1(\cos \theta)	&=&	-3\sin \theta\cos \theta\nonumber\\
P_2^2(\cos \theta)	&=&	3 \sin^2 \theta\nonumber\\
P_3^0(\cos \theta)	&=&	{1 \over 2}\cos \theta(5 \cos^2 \theta - 3)\nonumber\\
P_3^1(\cos \theta)	&=&	-{3 \over2} (5 \cos^2 \theta - 1)\sin \theta\nonumber\\
P_3^2(\cos \theta)	&=&	15\cos \theta \sin^2 \theta\nonumber\\
P_3^3(\cos \theta)	&=&	-15 \sin^3 \theta\nonumber\\
P_4^0(\cos\theta)	&=&	{1 \over 8}(35\cos\theta^4-30\cos\theta^2+3)\nonumber\\
P_4^1(\cos\theta)	&=&	{5 \over 2} \cos\theta(3-7\cos\theta^2)\sin\theta\nonumber\\
P_4^2(\cos\theta)	&=&	{15 \over 2} (7\cos\theta^2-1)\sin\theta^2\nonumber\\
P_4^3(\cos\theta)	&=&	-105 \cos\theta \sin\theta^3)\nonumber\\
P_4^4(\cos\theta)	&=&	105 \sin\theta^4\nonumber\\
P_5^0(\cos\theta)	&=&	{1 \over 8} \cos\theta(63\cos\theta^4-70\cos\theta^2+15)\nonumber\\
\end{eqnarray}

From this, we can calculate the values, as they are used in the code, which
does not include the comples exponential in the $\phi$ term. For example:\\

\begin{eqnarray}
YLM(2,3)&=&{1 \over \sqrt{4 \pi}} Y_4^2\nonumber\\
        &=&{1 \over \sqrt{4 \pi}}
	  \sqrt{{9 \over 4 \pi} {2! \over 6!}}
	  P_4^2(\cos\theta)\nonumber\\
	&=&{1 \over 8 \pi} \sqrt{1 \over 10}
	   ({15 \over 2} (7\cos\theta^2-1) \sin\theta^2)\nonumber\\
	&=&{3 \sqrt{10} \over 32 \pi} (7 \cos\theta^2 - 1) \sin\theta\nonumber\\
	&=&0.0943672 \times (7 * \cos\theta^2 -1) \sin\theta\nonumber\\
\end{eqnarray}

\section{YLM1}

\begin{eqnarray}
YLM1(2,1)&=&{1 \over \sqrt{4 \pi}} Y_1^0\nonumber\\
YLM1(2,2)&=&{1 \over \sqrt{4 \pi}} Y_1^1\nonumber\\
YLM1(3,1)&=&{1 \over \sqrt{4 \pi}} Y_2^0\nonumber\\
YLM1(3,2)&=&{1 \over \sqrt{4 \pi}} Y_2^1\nonumber\\
YLM1(3,3)&=&{1 \over \sqrt{4 \pi}} Y_2^2\nonumber\\
YLM1(4,1)&=&{1 \over \sqrt{4 \pi}} Y_3^0\nonumber\\
YLM1(4,2)&=&{1 \over \sqrt{4 \pi}} Y_3^1\nonumber\\
YLM1(4,3)&=&{1 \over \sqrt{4 \pi}} Y_3^2\nonumber\\
YLM1(4,4)&=&{1 \over \sqrt{4 \pi}} Y_3^3\nonumber\\
\end{eqnarray}

etc. etc.\\

Note therefore, that YLM1(3,1) = YLM(1,1) etc.\\

Again, we can calculate the values for YLM1 in the same way as before. e.g.
for YLM1(5,3), which is the same as YLM(2,3):\\

\begin{eqnarray}
YLM1(5,3)&=&{1 \over \sqrt{4 \pi}} Y_4^2\nonumber\\
 	 &=&{3 \sqrt{10} \over 32 \pi} (7 \cos\theta^2 - 1) \sin\theta\nonumber\\
         &=&{1 \over 4 \pi} {3 \sqrt{10} \over 8} (7 \cos\theta^2 - 1) \sin\theta\nonumber\\
\end{eqnarray}

Note that in the code, the ${1 \over 4 \pi}$ term is written as
0.0795774715, but the square root part is evaluated within the Fortran,
unlike the {\em YLM} function, where the whole thing is hard coded.\\

