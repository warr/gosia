\chapter{Coupling coefficients}
\label{chapt:coupling}

The coupling parameter $\zeta_{kn}^{(\lambda n)}$ is calculated in the
function {\em LSLOOP} and stored as {\em ZETA} in the common block {\em
CCOUP}. Note that other parameters are also stored in {\em ZETA} but the
real $\zeta_{kn}^{(\lambda n)}$ is at the start of this array.\\

\begin{equation}
\zeta_{kn}^{(\lambda n)} =
\sqrt{2 \lambda + 1}
(-1)^{I_n - M_n}
\big({I_n \over -M_n} {\lambda \over \mu} {I_k \over M_k}\big)
\psi_{kn}
\end{equation}

Note that in the code, ins = $2I_n$, lam2 = $2\lambda$, inr = $2I_k$, jg1 =
$-2M_n$, jg2 = $2\mu$ and jrmir = $2M_k$. The factors of two are needed
because these values are passed two the function {\em WTHREJ}, which
evaluates a Wigner 3-j symbol, as integers, so we need to double them to
allow for half integers. The function {\em WTHREJ} expects the values to be
doubled in this way.\\

Note also, that the $\sqrt{2 \lambda + 1}$ part is evaluated in the function
{\em LOAD} shortly before it calls {\em LSLOOP} and passed as a parameter.
This is because {\em LSLOOP} is called several times for the same value of
$\lambda$, so we only need to evaluate the square root once.\\

The coupling parameter $\psi_{kn}$ is calculated by the function {\em LOAD}
and stored in the variable {\em PSI} in the common block {\em PCOM}.\\

\begin{equation}
\psi_{kn} = 
C_\lambda^{E,M}
{Z \sqrt{A_1} \over (s Z_1 Z_2)^\lambda}
\{(E_p - s E_k)(E_p - s E_n)\}^{(2 \lambda - 1) / 4}
\end{equation}

The $C_\lambda^{E}$ coefficients are:\\

\begin{equation}
C_\lambda^E = 1.116547 \cdot (13.889122)^\lambda \cdot
{(\lambda - 1)! \over (2 \lambda + 1)!!}
\end{equation}

and the $C_\lambda^{M}$ coefficients are:\\

\begin{equation}
C_\lambda^M = {v \over c} \cdot {C_\lambda^E \over 95.0981942}
\end{equation}

\section{The code}

The function {\em LOAD} is the entry point. Here we calculate both $\xi$ and
$\psi$ and from here we call {\em LSLOOP} which calculates $\zeta$.\\

The function {\em LSLOOP}, in turn, calls {\em CODE7}, {\em LEADF} and {\em
WTHREJ}. Only {\em WTHREJ} is concerned with the physics (evaluating a
Wigner 3-j symbol) as the other two functions are used for indexing and
optimising purposes.\\

\begin{verbatim}
                 /- CODE7
LOAD --- LSLOOP {-- LEADF
                 \- WTHREJ
\end{verbatim}
