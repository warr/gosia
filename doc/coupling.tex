\chapter{Coupling coefficients}
\label{chapt:coupling}

The coupling parameter $\zeta_{kn}^{(\lambda n)}$ is calculated in the
function {\em LSLOOP} and stored as {\em ZETA} in the common block {\em
CCOUP}. Note that other parameters are also stored in {\em ZETA} but the
real $\zeta_{kn}^{(\lambda n)}$ is at the start of this array.\\

\begin{equation}
\zeta_{kn}^{(\lambda n)} =
\sqrt{2 \lambda + 1}
(-1)^{I_n - M_n}
\big({I_n \over -M_n} {\lambda \over \mu} {I_k \over M_k}\big)
\psi_{kn}
\end{equation}

Note that in the code, ins = $2I_n$, lam2 = $2\lambda$, inr = $2I_k$, jg1 =
$-2M_n$, jg2 = $2\mu$ and jrmir = $2M_k$. The factors of two are needed
because these values are passed two the function {\em WTHREJ}, which
evaluates a Wigner 3-j symbol, as integers, so we need to double them to
allow for half integers. The function {\em WTHREJ} expects the values to be
doubled in this way.\\

Note also, that the $\sqrt{2 \lambda + 1}$ part is evaluated in the function
{\em LOAD} shortly before it calls {\em LSLOOP} and passed as a parameter.
This is because {\em LSLOOP} is called several times for the same value of
$\lambda$, so we only need to evaluate the square root once.\\

The coupling parameter $\psi_{kn}$ is calculated by the function {\em LOAD}
and stored in the variable {\em PSI} in the common block {\em PCOM}.\\

\begin{equation}
\psi_{kn} = 
C_\lambda^{E,M}
{Z \sqrt{A_1} \over (s Z_1 Z_2)^\lambda}
\{(E_p - s E_k)(E_p - s E_n)\}^{(2 \lambda - 1) / 4}
\end{equation}

The $C_\lambda^{E}$ coefficients are:\\

\begin{equation}
C_\lambda^E = 1.116547 \cdot (13.889122)^\lambda \cdot
{(\lambda - 1)! \over (2 \lambda + 1)!!}
\end{equation}

and the $C_\lambda^{M}$ coefficients are:\\

\begin{equation}
C_\lambda^M = {v \over c} \cdot {C_\lambda^E \over 95.0981942}
\end{equation}

\section{Theory}

Alder, Bohr, Huus, Mottelson and Winther give\footnote{In the book Coulomb
Excitation edited by Alder and Winther}:

\begin{equation}
C_{E\lambda} = { Z_1^2 \over 40.03} [0.07199(1+A_1/A_2) Z_1 Z_2]^{-2\lambda+2}
\end{equation}

However, the $\xi$ parameter has an additional factor of f$_\lambda$ = ${(\lambda - 1)!
\over (2 \lambda + 1)!!}$, which Gosia includes in the cpsi parmeters in the
function {\em LOAD}.\\

\begin{tabular}{l|l|l|l|l}
\hline
$\lambda$ & $f_\lambda$ & Gosia factor & Gosia
factor /  $f_\lambda$ & Winther de Boer\\
\hline
E1 & 1/3   & 5.169286   & 15.507828 & 0.0249812\\
E2 & 1/15  & 14.359366  & 215.39049 & 4.8202555\\
E3 & 2/105 & 56.982577  & 2991.5853 & 930.09159\\
E4 & 2/315 & 263.812652 & 41550.492 & 179465.66\\
E5 & 24/10394 & 1332.40950 & 577099.86 & 32628766\\
\end{tabular}

The Gosia factor / $f_\lambda$ changes each time by a factor of 0.07199. i.e.

\begin{equation}
f_{\mathrm{gosia}} = 1.116547
{(\lambda - 1)! \over (2 \lambda + 1)!!}
(13.889122)^{\lambda}
\end{equation}

where 13.889112 is the reciprocal of 0.07199.

\begin{equation}
{2 \pi^2 \sqrt{2} \over 25} = 0.1166183
\end{equation}

\begin{equation}
{125 \over 9} = 13.88888889
\end{equation}

The Winther de Boer factor changes by (0.07199)$^2$, as can be seen from the
equation above.

For the magnetic case, we have

\begin{equation}
C_\lambda^M = (v/c) C_\lambda^E / 95.0981942.
\end{equation}

\begin{equation}
{\hbar c \over e^4} = {197.327 \over 1.439976^2} = 95.1647
\end{equation}

In fact, the Gosia $cpsi$ is actually C$_\lambda$ / $(Z_1 Z_2 (1 +
A_1/A_2)^\lambda$.

\section{The code}

The function {\em LOAD} is the entry point. Here we calculate both $\xi$ and
$\psi$ and from here we call {\em LSLOOP} which calculates $\zeta$.\\

The function {\em LSLOOP}, in turn, calls {\em CODE7}, {\em LEADF} and {\em
WTHREJ}. Only {\em WTHREJ} is concerned with the physics (evaluating a
Wigner 3-j symbol) as the other two functions are used for indexing and
optimising purposes.\\

\begin{verbatim}
                 /- CODE7
LOAD --- LSLOOP {-- LEADF
                 \- WTHREJ
\end{verbatim}
