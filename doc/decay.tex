\section{DECAY}
\label{sect:decay}

\noindent This function calculates the decay statistical tensor describing
the mixed electric and magnetic transition from a state $I$ to a state
$I_f$:\\

\begin{equation}
R_{k\chi}(I, I_f) = {1 \over 2 \gamma(I) \sqrt{\pi}}
G_k \rho_{k\chi}
\sum_{\lambda \lambda^\prime}
\delta_\lambda \delta_{\lambda^\prime}^* F_k(\lambda \lambda^\prime I_f I)
\end{equation}

\noindent The Coulomb excitation statistical tensors $\rho$ are purely real
and the selection rules for electromagnetic transitions imply that the
product $\delta_\lambda \delta_{\lambda^\prime}^*$ is also real.
Consequently, the decay statistical tensors are also real.\\

\noindent In this function {\em ZETA} does not contain $\zeta$ but instead
has the $\rho$ statistical tensors, rotated into the new frame of reference.\\

\noindent It uses the function {\em GKVAC} to calculate the nuclear
deorientation due to vacuum polarization. This sets the {\em VACDP}
coefficients which are time-dependent nuclear deorientation coefficients
$G_k(t)$ for the given level.\\

\noindent We also modify the statistical tensors to account for multiple
excitation.\\

\begin{equation}
R_{k\chi}(I, I_f) \rightarrow R_{k\chi}(I, I_f) +
\sum_n R_{k\chi}(I_n, I) H_k(I, I_n)
\end{equation}

\noindent where the $H_k$ values were calculated from {\em SEQ} (see section
\ref{sect:gf}) by calling {\em GF} (see section \ref{sect:gf}), where they
are stored as {\em GKP}.\\

\begin{equation}
H_k(I, I_n) =
{\sqrt{(2 I + 1) (2 I_n + 1)} \over \gamma(I)}
\sum_\lambda (-1)^{I + I_n + \lambda + k}
|\delta_\lambda|^2
(1 + c(\lambda))
\begin{Bmatrix}
I & I & k\\
I_n & I_n & \lambda
\end{Bmatrix}
\end{equation}

\noindent For each experimental lifetime provided by the user, we compare
here the calculated one and add the appropriate amount to $\chi^2$ and the
log of $\chi^2$.\\

