\section{GKK}
\label{sect:gkk}

\noindent This function calculates the time-dependent deorientation
coefficients.\\

\noindent First we call {\em XSTATIC} (see section
\ref{sect:xstatic}) to calculate the static part. This calculates
QCEN (the centre of the gaussian charge state distribution), DQ (the
gaussian width of this distribution and XNOR (the normalisation
parameter needed to ensure that the sum over probabilities is unity.\\

\begin{equation}
<a_k> = \sum_l p(J_1) \sum_F {(2 F + 1)^2 \over 2 J_1 + 1} \times
\begin{Bmatrix}
F & F & k\\
I & I & J_1
\end{Bmatrix}
^2
\end{equation}

\noindent The function {\em WSIXJ} (see section \ref{sect:wsixj}) is
used to evaluate the Wigner 6-j symbol. Note that the numbers passed
to this function are integers which are twice the value of the
spins, so we are usually dealing with spin values that have been
doubled.\\


\noindent Note that originally, the function {\em ATS} was used to determine
the ground-state atomic spin for an atom with $N$ electrons. This was
implemented as a complicated set of nested if statements. This has been
replaced with a simple table which is included in this function.\\

