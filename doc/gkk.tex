\section{GKK}
\label{sect:gkk}

\noindent This function calculates the time-dependent deorientation
coefficients.\\

\noindent First we call {\em XSTATIC} (see section \ref{sect:xstatic}) to
calculate the static part. This calculates QCEN (the centre of the gaussian
charge state distribution), DQ (the gaussian width of this distribution and
XNOR (the normalisation parameter needed to ensure that the sum over
probabilities is unity.\\

\begin{equation}
<a_k> = \sum_l p(J_1) \sum_F {(2 F + 1)^2 \over 2 J_1 + 1} \times
\begin{Bmatrix}
F & F & k\\
I & I & J_1
\end{Bmatrix}
^2
\end{equation}

\noindent where $F$ is the vector sum of the nuclear spin $I$ and the
electronic spin $J_1$, which is the ground-state spin of the dexcited atom. 
Note that originally, the function {\em ATS} was used to determine the
ground-state atomic spin for an atom with $N$ electrons. This was
implemented as a complicated set of nested if statements. This has been
replaced with a simple table which is included in this function.\\

\noindent The function {\em WSIXJ} (see section \ref{sect:wsixj}) is
used to evaluate the Wigner 6-j symbol. Note that the numbers passed
to this function are integers which are twice the value of the
spins, so we are usually dealing with spin values that have been
doubled.\\

\begin{equation}
\bar{H} = K Z \bigg({v \over c}\bigg)^x
\end{equation}

\noindent where $\bar{H}$ is {\em hmean} in the code. The parameters $x$ is
{\em POWER} and $K$ is {\em FIEL}, which have the default values of 0.7 and
6 $\times$ $10^{-6}$, respectively, but these values may be adjusted by the
user using the CONT option ``VAC,''.\\

\begin{equation}
a(J_i) = \mu_n g {\bar{H} \over J_i}
\end{equation}

\noindent where $a(J_i$) is called {\em wsp} (but only temporarily), $g$ is
{\em GFAC} and $J_i$ is {\em AVIJ}. The value of $\mu_n$ is hardcoded as
4789. {\em GFAC} is another parameter that can be set by the user via
``VAC,''. Its default value is $Z \over A$.\\

\begin{equation}
<\omega^2> = {1 \over 3} k (k + 1) \sum_{J_i} p(J_i) {a^2(J_i) \over
\hbar^2} J_i (J_i + 1)
\end{equation}

\noindent For k = 1, 2, 3 we calculate:\\

\begin{equation}
\lambda_k = { 1 - <\alpha_k> \over \tau_C}
\bigg(1 - e^{- <\omega^2> \tau_C^2\over 1 - <\alpha_k>}\bigg)
\end{equation}

and store it as {\em xlam}.\\

\noindent Next we calculate:\\

\begin{equation}
G^{fluct}(t) = (1 - \lambda_k \tau_C) e^{-\lambda_k t}
\end{equation}

\noindent which we store as {\em GKI}.\\

\noindent In order to allow for lifetimes of comparable order to $\tau_C$,
we calculate:\\

\begin{equation}
r = (\Lambda^* + \lambda_k) \tau_I + 1
\end{equation}

\noindent which is stored in {\em down}. $\tau_I$ is {\em Time}, the
lifetime of the state. $\Lambda^*$ is {\em XLAMB}.\\

\noindent Then we calculate:\\

\begin{equation}
p = {\sqrt{9 \lambda_k^2 + 8\lambda_k\tau_C (<\omega^2> - \lambda_k^2)} \over
4 \lambda_k \tau_C}
\end{equation}

\noindent where $p$ is called {\em alp} and $\lambda_k$ is {\em xlam} and
$tau_C$ is {\em TIMEC} which is another parameter set via ``VAC,'' and which
has a default value of 3.5.\\

\noindent Then we can calculate:\\

\begin{equation}
G_k = \lambda_I \int_0^\infty G_k(t) e^{-\lambda_I t} dt =
G_k^{BS} \bigg[ 1 + 
{\lambda_k \tau_I (r - 2 p^2 \tau_I \tau_C \over
(r + p \tau_I)(r + 2 p \tau_I)}\bigg]
\end{equation}

\noindent where we first calculate the numerator in {\em upc} and then the
denominator in {\em dwc}. This gives us a correction factor which we
multiply with {\em GKI} and store back in {\em GKI}.\\
