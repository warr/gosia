\section{ANGULA}
\label{sect:angula}

\noindent This function calculates the angular distribution of the emitted
$\gamma$ rays.\\

\begin{equation}
{d^2\omega \over d\Omega_p d\Omega_\gamma} =
\sigma_R(\theta_p)
{1 \over 2\gamma(I_1) \sqrt{\pi}}
\sum_{k even, \chi}
\rho_{k\chi}^*(I_1)
\sum\_{\lambda\lambda^\prime}
\delta_\lambda
\delta_{\lambda^\prime}^*
F_k(\lambda \lambda^\prime I_2 I_1)
Y_{k\chi}(\theta_\gamma, \phi_\gamma)
\end{equation}

\noindent with:\\

\begin{equation}
\gamma(I_1) = \sum_{\lambda n} |\delta_\lambda(I_1 \rightarrow I_n)|^2
\end{equation}

\noindent We loop over the decays in {\em KSEQ} which are ordered
starting from the top of the level scheme working down. Then we calculate
the decay between each pair of levels, using the F coefficients and $\delta$
values together with the statistical tensors, which are stored at the
beginning of the ZETA block.\\


\noindent The function {\em RECOIL} is used to perform a relativistic
correction (see section \ref{sect:recoil}). The function {\em FIINT1} (see
section \ref{sect:fiint1}) is used to integrate over $\phi$ in the lab
frame, while {\em FIINT} (see section \ref{sect:fiint}) is used to do that
in the frame of the recoiling nucleus. The functions {\em YLM} and {\em
YLM1} (see sections \ref{sect:ylm} and \ref{sect:ylm}) are used to calculate
the spherical harmonics.\\

