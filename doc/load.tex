\section{LOAD}
\label{sect:load}

The function load is used to load the coupling parameters $\xi$ and $\psi$
into arrays. It also calls {\em LSLOOP} (see section \ref{sect:lsloop}) to
caclulate the $\zeta$ coupling parameters.\\

\noindent It is also in this function that we deal with the E1 polarisation from the
GDR. For target excitation we calculate the value zpol as:

\begin{equation}
\mathrm{zpol} = \mathrm{dipol} \times E_p {A_2 \over(Z_2^2 (1 + {A_1 \over
A_2}))}
\end{equation}

\noindent where dipol is a parameter, which can be changed by the user but is 0.005 by
default.\\

\noindent For projectile excitation we use the following formula instead:\\

\begin{equation}
\mathrm{zpol} = \mathrm{dipol} \times E_p {A_2 \over(Z_2^2 (1 + {A_2 \over
A_1}))}
\end{equation}

\noindent The calculation of $\xi$ uses:\\

\begin{equation}
\eta = {Z_1 Z_2 \sqrt{A_1 \over E_p} \over 6.349770}
\end{equation}

\noindent where the value 6.349770 comes from:\\

\begin{equation}
6.349770 = {197.33 \over 1.44} \sqrt{2 \over 931.49}
\end{equation}

\noindent where the atomic mass unit is 931.494028 MeV/$c^2$, e$^2$ =
1.439976 MeV fm and $\hbar c$ = 197.3269631 MeV fm\\

\noindent then we have for each level:\\

\begin{equation}
\mathrm{dep} = \bigg( 1 + {A_1 \over A_2} \bigg) E_n
\end{equation}

\noindent where $E_n$ is the excitation energy of the level.\\

\begin{equation}
\mathrm{zet} = {\mathrm{dep} \over E_p}
\end{equation}

\noindent where $E_p$ is the projectile energy.\\

\begin{equation}
\mathrm{szet} = \sqrt{1 - \mathrm{zet}}
\end{equation}

\begin{equation}
\mathrm{etan} = {\eta \over \mathrm{szet}}
\end{equation}

\noindent The value etan is calculated for each level and $\xi$ for each
matrix element is simply the difference between the values of etan for the
two levels connected by that matrix element.\\

\noindent We also calculate ${C_\lambda \over (s Z_1 Z_2)^\lambda}$:\\

\noindent where s is $1 + {A_1 \over A_2}$\\

\begin{equation}
\mathrm{aazz} = {1 \over {(1 + {A_1 \over A_2)}} Z_1 Z_2}
\end{equation}

\begin{equation}
\mathrm{cspi}_1 = {C_1^E \over s Z_1 Z_2} = 5.169286 ~\mathrm{aazz}
\end{equation}

\begin{equation}
\mathrm{cspi}_2 = {C_2^E \over (s Z_1 Z_2)^2} = 14.359366 ~\mathrm{aazz}^2
\end{equation}

\begin{equation}
\mathrm{cspi}_3 = {C_3^E \over (s Z_1 Z_2)^3} = 56.982577 ~\mathrm{aazz}^3
\end{equation}

\begin{equation}
\mathrm{cspi}_4 = {C_2^E \over (s Z_1 Z_2)^4} = 263.812653 ~\mathrm{aazz}^4
\end{equation}

\begin{equation}
\mathrm{cspi}_5 = {C_2^E \over (s Z_1 Z_2)^5} = 1332.409500 ~\mathrm{aazz}^5
\end{equation}

\begin{equation}
\mathrm{cspi}_6 = {C_2^E \over (s Z_1 Z_2)^6} = 7117.691577 ~\mathrm{aazz}^6
\end{equation}

\noindent and for the magnetic excitations:\\

\begin{equation}
\mathrm{aazz} = {v_\infty \over 95.0981942}
\end{equation}

\noindent where the value of 95.0981942 is approximately 137.03599 /
1.4399645 and the former is the inverse fine structure constant and the
latter is $e^2$ in MeV fm.\\

\begin{equation}
\mathrm{cspi}_7 = {C_1^M \over s Z_1 Z_2} = \mathrm{cspi}_1 ~\mathrm{aazz}
\end{equation}

\begin{equation}
\mathrm{cspi}_8 = {C_2^M \over (s Z_1 Z_2)^2} = \mathrm{cspi}_2 ~\mathrm{aazz}^2
\end{equation}

\noindent Next the function {\em LOAD} sets the values of $\psi$ for each
matrix element and each multipolarity:\\

\begin{equation}
\psi = C_\lambda Z_1 \sqrt{A_1} \large[(E_p - (1 + {A_1 \over A_2}) E_1)
(E_p - (1 + {A_1 \over A_2}) E_2)\large]^{({2 \lambda - 1 \over 4})}
\end{equation}

\noindent where $E_p$ is the projectile energy and $E_1$ and $E_2$ are the
energies of the states connected by this matrix element.\\

\noindent After that, the NSTART, NSTOP and CAT arrays are set up. There is
a CAT entry for each substate and it has three values: the index of the
level, the spin of that level and the $m$ quantum number of the substate.
The NSTART and NSTOP arrays indicate the first and last CAT entry for each
level.\\

\noindent The LZETA array which gives the start of the $\zeta$ coupling
coefficients for each multipolarity is initialized to zero and then for each
multipolarity, we loop over the substates calling {\em LSLOOP} (see section
\ref{sect:lsloop}) to calculate the $\zeta$ coefficients.\\

\noindent Finally, we make a check that we haven't exceeded the limit for
the number of $\zeta$ coefficients. This made sense when the collision
coefficients came after the $\zeta$ values in the same ZETA array. Then it
was possible to overwrite the collision coefficients without exceeding the
bounds of the array. Since they were separated out, this test is pretty
useless if we have bounds checking enabled, because it would have already
thrown an exception for exceeding the bounds before this test. Also, it only
gives a warning, and probably Gosia's memory is corrupted!\\
