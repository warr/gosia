\chapter{Efficiency calibrations in gosia}
\label{chapt:efficiency}

There are five methods which can be used for efficiency calibrations:\\

\begin{itemize}
\item 0. Gremlin
\item 1. Jaeri
\item 2. Fiteff
\item 3. Leuven
\item 4. Radware
\end{itemize}

\section{Gremlin efficiency calibration}

Parameters are: $a_0$, $a_1$, $a_2$, $a_3$, $f$, $N$, $b$, $c$. Note that b
and c have units of keV. The energy of the $\gamma$ ray is $E_n$ (in keV -
note that Gosia works in MeV, so this has to be converted) and the resulting
efficiency is $\epsilon$. The value of $f$ must be negative or zero.

\begin{equation}
w = \ln\big({E_n \over E_0}\big)
\end{equation}

where $E_0$ = 50 keV.

\begin{equation}
\ln(\epsilon_P) = \sum_{k=0,3} a_k \cdot w^k
\end{equation}

If f $<$ 0, we have an additional F-factor:

where N is a positive integer.

\begin{equation}
\ln(\epsilon_F) = f w^{-N}
\end{equation}

If c $>$ 0, we have an additional Woods-Saxon term:

\begin{equation}
r = {b - E_n \over c}
\end{equation}

\begin{equation}
\epsilon_{ws} = {1 \over (1 + e^r)}
\end{equation}

The final resulting efficiency is the product of the three terms:

\begin{equation}
\epsilon = \epsilon_P \cdot \epsilon_F \cdot \epsilon_{ws}
\end{equation}

\section{Jaeri efficiency calibration}

Parameters are $a_0$, $a_1$, $a_2$, $a_3$, $f$. The energy of the $\gamma$
ray is $E_n$ (in keV - note that Gosia works in MeV, so this has to be
converted) and the resulting efficiency is $\epsilon$. The value of $f$ is
not used, but it must be $>$ 10 to be sure that this code is called.

\begin{equation}
w = \ln\big({E_n \over 511}\big)
\end{equation}

\begin{equation}
\ln(\epsilon) = a_0 + a_1 \cdot w - e^{(a_2 + a_3 \cdot w)}
\end{equation}

\section{Fiteff efficieny calibration}

Parameters are $a_0$, $a_1$, $a_2$, $a_3$, $f$. The energy of the $\gamma$
ray is $E_n$ (in keV - note that Gosia works in MeV, so this has to be
converted) and the resulting efficiency is $\epsilon$. The value of $f$ must
be $>$ 0 and $<$ 10 to be sure that this code is called.

\begin{equation}
w = \ln\big({E_n \over 1000 f}\big)
\end{equation}

If $E_n$ $<$ 1000 f:

\begin{equation}
\ln(\epsilon) = a_0 + a_1 \cdot w + w^2 \cdot (a_2 + w \cdot a_3)
\end{equation}

Otherwise:

\begin{equation}
\ln(\epsilon) = a_0 + a_1 \cdot w
\end{equation}

\section{Leuven efficiency calibration}

\begin{equation}
w_1 = \ln(E_n)
\end{equation}

\begin{equation}
\epsilon = \sum_{k=0,5} a_k \cdot w^k
\end{equation}


\section{Radware efficiency calibration}

Parameters are $a_0$, $A$, $B$, $C$, $D$, $E$, $F$, $G$. The
energy of the $\gamma$ ray is $E_n$ (in keV - note that Gosia works in MeV,
so this has to be converted) and the resulting efficiency is $\epsilon$.
This method can only be turned on by using the EFF, option of CONT.

\begin{equation}
x = \ln\big({E_n/100}\big)
\end{equation}

\begin{equation}
\epsilon_1 = (A + B \cdot x + C \cdot x^2) ^{-G}
\end{equation}

\begin{equation}
y = \ln(E_n/1000)
\end{equation}

\begin{equation}
\epsilon_2 = (D + E \cdot y + F \cdot y^2)^{-G}
\end{equation}

\begin{equation}
\ln(\epsilon) = a_0 + ((A + B \cdot x + C \cdot x^2) ^{-G} +
(D + E \cdot y + F \cdot y^2)^{-G})^{-{1 \over G}}
\end{equation}

Ref: Nuclear Instruments and Methods in physics Research A 361 (1995)
297-305, but with the addition of the $a_0$ term, which is not in the NIM
article.
