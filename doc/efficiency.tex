\chapter{Efficiency calibrations in gosia}
\label{chapt:efficiency}

There are five methods which can be used for efficiency calibrations:\\

\begin{itemize}
\item 0. Gremlin
\item 1. Jaeri
\item 2. Fiteff
\item 3. Leuven
\item 4. Radware
\end{itemize}

\section {Using CONT option EFF,}
The option EFF, in CONT is used as follows:
\begin{verbatim}
EFF,3.
1,4
4,4
8,4
\end{verbatim}

where the number ``3.'' on the EFF, line indicates that there are three
subsequent lines, each of which has the format ``Experiment number,
efficiency calibration number''. In this case, we set experiments 1, 4 and 8
to use the Radware calibration.

%%%%%%%%%%%%%%%%%%%%%%%%%%%%%%%%%%%%%%%%%%%%%%%%%%%%%%%%%%%%%%%%%%%%%%%%%%%%%%
\section{Selecting a calibration}

The OP,RAW input takes eight parameters A1{\ldots} A8. Note that where the
parameters correspond to energies, they are in keV, but gosia uses MeV, so
there are factors of 1000 in the code to convert.

If A8 is less than -999, or we selected calibration 3 in CONT option EFF,
then we use the Leuven calibration.

If we selected calibration 4 in CONT option EFF, then we use the Radware
calibration.

If A5 is greater than zero but less than ten, or we selected calibration 2
in CONT option EFF, then we use the Fiteff calibration.

If A5 is less than ten and we didn't select calibration 1 in the CONT option
EFF, then we use the Gremlin calibration.

Otherwise, we use the Jaeri calibration.

\begin{verbatim}
      IF ( (AKAVKA(8,Ipd).LE.-999.) .OR. (AKAVKA(9,Ipd).EQ.3.) ) THEN
         GOTO 1003 ! Leuven
      ELSEIF ( AKAVKA(9,Ipd).EQ.4. ) THEN
         GOTO 1004 ! Radware
      ELSEIF ( (AKAVKA(5,Ipd).GT.0. .AND. AKAVKA(5,Ipd).LT.10.) .OR. 
     &         (AKAVKA(9,Ipd).EQ.2.) ) THEN
         GOTO 1002 ! Fiteff
      ELSEIF ( (AKAVKA(5,Ipd).LT.10.) .AND. (AKAVKA(9,Ipd).NE.1.) ) THEN
         GOTO 1000 ! Gremlin
      ENDIF
      GOTO 1001 ! Jaeri
\end{verbatim}

where AKAVKA(1{\ldots}8,Ipd) are the parameters A1{\ldots}8 and
AKAVKA(9,Ipd) is the value set for the experiment in the EFF, option of CONT.

%%%%%%%%%%%%%%%%%%%%%%%%%%%%%%%%%%%%%%%%%%%%%%%%%%%%%%%%%%%%%%%%%%%%%%%%%%%%%%
\section{Gremlin efficiency calibration}

Parameters are: $a_1$, $a_2$, $a_3$, $a_4$, $f$, $N$, $b$, $c$. Note that b
and c have units of keV. The energy of the $\gamma$ ray is $E_n$ (in keV -
note that Gosia works in MeV, so this has to be converted) and the resulting
efficiency is $\epsilon$. The value of $f$ must be negative or zero.

\begin{equation}
w = \ln\big({E_n \over E_0}\big)
\end{equation}

where $E_0$ = 50 keV.

\begin{equation}
\ln(\epsilon_P) = \sum_{k=1,4} a_k \cdot w^k
\end{equation}

If f is non-zero (note that it should be either zero or negative), we have
an additional F-factor:

\begin{equation}
\ln(\epsilon_F) = f w^{-N}
\end{equation}

where N is a positive integer.\\

If c $>$ 0, we have an additional Woods-Saxon term:

\begin{equation}
r = {b - E_n \over c}
\end{equation}

\begin{equation}
\epsilon_{ws} = {1 \over (1 + e^r)}
\end{equation}

The final resulting efficiency is the product of the three terms:

\begin{equation}
\epsilon = \epsilon_P \cdot \epsilon_F \cdot \epsilon_{ws}
\end{equation}

\begin{verbatim}
 1000 w = LOG(20.*En) ! E0 = 50 keV, so w = LOG(En/E0) with En in MeV
      pw = AKAVKA(1,Ipd) + AKAVKA(2,Ipd)*w + AKAVKA(3,Ipd)
     &     *w*w + AKAVKA(4,Ipd)*w*w*w
      Effi = Effi*EXP(pw)
      IF ( ABS(AKAVKA(5,Ipd)).GE.1.E-9 ) THEN ! F-factor
         n = INT(AKAVKA(6,Ipd)+.1)
         pw = w**n
         w = AKAVKA(5,Ipd)/pw
         Effi = Effi*EXP(w)
      ENDIF
      IF ( ABS(AKAVKA(8,Ipd)).LT.1.E-9 ) RETURN
      w = (AKAVKA(7,Ipd)-1000.*En)/AKAVKA(8,Ipd) ! Woods-saxon factor
      pw = EXP(w)
      IF ( ABS(pw-1.).LT.1.E-6 ) WRITE (22,99001)
99001 FORMAT (5x,'***** CRASH - EFFIX *****')
      Effi = Effi/(1.+pw) ! Older versions of gosia have a minus sign here, which is wrong
                          ! because it is not what is done in gremlin (FITFUN) or the gosia manual
      RETURN
\end{verbatim}

%%%%%%%%%%%%%%%%%%%%%%%%%%%%%%%%%%%%%%%%%%%%%%%%%%%%%%%%%%%%%%%%%%%%%%%%%%%%%%
\section{Jaeri efficiency calibration}

Parameters are $a_1$, $a_2$, $a_3$, $a_4$, $f$. The energy of the $\gamma$
ray is $E_n$ (in keV - note that Gosia works in MeV, so this has to be
converted) and the resulting efficiency is $\epsilon$. The value of $f$ is
not used, but it must be $>$ 10 to be sure that this code is called.

\begin{equation}
w = \ln\big({E_n \over 511}\big)
\end{equation}

\begin{equation}
\ln(\epsilon) = a_1 + a_2 \cdot w - e^{(a_3 + a_4 \cdot w)}
\end{equation}

\begin{verbatim}
 1001 w = LOG(En/.511)
      Effi = EXP(AKAVKA(1,Ipd)+AKAVKA(2,Ipd)
     &       *w-EXP(AKAVKA(3,Ipd)+AKAVKA(4,Ipd)*w))
      RETURN
\end{verbatim}

%%%%%%%%%%%%%%%%%%%%%%%%%%%%%%%%%%%%%%%%%%%%%%%%%%%%%%%%%%%%%%%%%%%%%%%%%%%%%%
\section{Fiteff efficieny calibration}

Parameters are $a_1$, $a_2$, $a_3$, $a_3$, $f$. The energy of the $\gamma$
ray is $E_n$ (in keV - note that Gosia works in MeV, so this has to be
converted) and the resulting efficiency is $\epsilon$. The value of $f$ must
be $>$ 0 and $<$ 10 to be sure that this code is called.

\begin{equation}
w = \ln\big({E_n \over 1000 f}\big)
\end{equation}

If $E_n$ $<$ 1000 f:

\begin{equation}
\ln(\epsilon) = a_1 + a_2 \cdot w + w^2 \cdot (a_3 + w \cdot a_4)
\end{equation}

Otherwise:

\begin{equation}
\ln(\epsilon) = a_0 + a_1 \cdot w
\end{equation}

\begin{verbatim}
 1002 w = LOG(En/AKAVKA(5,Ipd))
      pw = AKAVKA(2,Ipd)*w
      IF ( En.LT.AKAVKA(5,Ipd) ) pw = pw +
     &     w*w*(AKAVKA(3,Ipd)+w*AKAVKA(4,Ipd))
      Effi = Effi*EXP(pw)*AKAVKA(1,Ipd)
      RETURN
\end{verbatim}

%%%%%%%%%%%%%%%%%%%%%%%%%%%%%%%%%%%%%%%%%%%%%%%%%%%%%%%%%%%%%%%%%%%%%%%%%%%%%%
\section{Leuven efficiency calibration}

Parameters are $a_1$, $a_2$, $a_3$, $a_4$, $a_5$, $a_6$.

\begin{equation}
w = \ln(E_n)
\end{equation}

\begin{equation}
\ln(\epsilon) = \sum_{k=1,6} a_k \cdot w^k
\end{equation}


\begin{verbatim}
 1003 Effi = AKAVKA(1,Ipd)
      w = LOG(1000.*En)
      DO i = 1 , 6
         Effi = Effi + AKAVKA(i+1,Ipd)*w**i
      ENDDO
      Effi = EXP(Effi)
      RETURN
\end{verbatim}

%%%%%%%%%%%%%%%%%%%%%%%%%%%%%%%%%%%%%%%%%%%%%%%%%%%%%%%%%%%%%%%%%%%%%%%%%%%%%%
\section{Radware efficiency calibration}

Parameters are $a_1$, $A$, $B$, $C$, $D$, $E$, $F$, $G$. The
energy of the $\gamma$ ray is $E_n$ (in keV - note that Gosia works in MeV,
so this has to be converted) and the resulting efficiency is $\epsilon$.
This method can only be turned on by using the EFF, option of CONT.

\begin{equation}
x = \ln\big({E_n/100}\big)
\end{equation}

\begin{equation}
y = \ln(E_n/1000)
\end{equation}

\begin{equation}
\ln(\epsilon) = a_1 + ((A + B \cdot x + C \cdot x^2) ^{-G} +
(D + E \cdot y + F \cdot y^2)^{-G})^{-{1 \over G}}
\end{equation}

Ref: Nuclear Instruments and Methods in physics Research A 361 (1995)
297-305, but with the addition of the $a_1$ term, which is not in the NIM
article.

\begin{verbatim}
 1004 w = LOG(En/.1)
      Effi = (AKAVKA(2,Ipd)+(AKAVKA(3,Ipd)+AKAVKA(4,Ipd)*w)*w)
     &       **(-AKAVKA(8,Ipd))
      w = LOG(En)
      Effi = (AKAVKA(5,Ipd)+(AKAVKA(6,Ipd)+AKAVKA(7,Ipd)*w)*w)
     &       **(-AKAVKA(8,Ipd)) + Effi
      Effi = AKAVKA(1,Ipd)*EXP(Effi**(-1/AKAVKA(8,Ipd)))
      RETURN
\end{verbatim}

