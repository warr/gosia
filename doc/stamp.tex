\section{STAMP}
\label{sect:stamp}

\noindent This function estimates the amplitude by approximating the Coulomb
interaction strengths using sine and cosine integrals. These are evaluated by
{\em TRINT} (see section \ref{sect:trint}).\\

\noindent These are effective strength parameters, which are only functions
of $\epsilon$ and $\xi$, so for a given experiment, assuming the
eccentricity $\epsilon$ to be constant, we only need to evaluate them as a
function of $\xi$.\\

\noindent For E1, the real and imaginary parts (apart from a factor) are:

\begin{equation}
\mathrm{cic} = \int_a^b {\cos(t) \over t^2} dt
\end{equation}

\begin{equation}
\mathrm{cis} = \int_a^b {\sin(t) \over t^2} dt
\end{equation}

\noindent respectively, where:\\

\begin{equation}
a = |\xi| ({\epsilon \over 2} e^{(\omega_0 - 3 d\omega)} + \omega_0 - 3
d\omega)
\end{equation}

\noindent and:\\

\begin{equation}
b = |\xi| ({\epsilon \over 2} e^{\omega_0} + \omega_0)
\end{equation}

\noindent Integration by parts gives us:\\

\begin{equation}
\mathrm{cic} = \int_a^b {\cos(t) \over t^2} dt = -\big[{\cos(t) \over t} \big]_a^b
- \int_a^b {\sin(t) \over t} dt
\end{equation}

\noindent and:\\

\begin{equation}
\mathrm{cis} = \int_a^b {\sin(t) \over t^2} dt = -\big[{\sin(t) \over t} \big]_a^b
+ \int_a^b {\cos(t) \over t} dt
\end{equation}

\noindent so we can express this integral in terms of ${\sin(a) \over a}$,
${\cos(a) \over a}$, ${\sin(b) \over b}$, ${\cos(b) \over b}$, and the sine
and cosine integrals from 0 to a and 0 to b, which are calculated using {\em
TRINT} (see section \ref{sect:trint}).\\

\noindent For E2 and M1 the real and imaginary parts are:

\begin{equation}
\mathrm{dwi} = {1 \over 6} \int_a^b {\cos(t) \over t^3} dt =
- {1 \over 2} \big[{\cos(t) \over t^2} \big]_a^b
-  {1 \over 2} \int_a^b {\sin(t) \over t^2} dt
\end{equation}

\noindent and:\\

\begin{equation}
\mathrm{exa} = {1 \over 6} \int_a^b {\sin(t) \over t^3} dt =
-{1 \over 2} \big[{\sin(t) \over t^2} \big]_a^b
+ {1 \over 2} \int_a^b {\cos(t) \over t^2} dt
\end{equation}

\noindent respectively.\\

\noindent These values are multiplied by factors and the complex conjugate
is returned.\\

\begin{center}
\begin{tabular}{|llll|}
\hline
E/M & $\lambda$ & $\mu$ & Factor\\
\hline
E & 1 & 0 & ${|\xi| \over 2 \epsilon}$\\
E & 1 & $\pm$1 & ${1\over 2}\sqrt{1 \over 2} |\xi| {\sqrt{\epsilon^2 - 1} 
\over \epsilon}$\\
E & 2 & 0 & ${3 \over 4} \xi^2 {(3 - \epsilon^2) \over \epsilon^2}$\\
E & 2 & $\pm$1 & ${3 \over 2} \sqrt{3 \over 2} \xi^2 {\sqrt{\epsilon^2 - 1} \over
\epsilon^2}$\\
E & 2 & $\pm$2 & ${3 \over 4} \sqrt{3 \over 2} \xi^2 {(\epsilon^2 - 1) \over
\epsilon^2}$\\
M & 1 & 0 & 0\\
M & 1 & $\pm$ 1 & ${\sqrt{2} \over 4} \xi^2 \sqrt{\epsilon^2 - 1}$\\
\hline
\end{tabular}
\end{center}
