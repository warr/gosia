\section{GF}
\label{sect:gf}

\noindent This function calculates the $H_k$ coefficients to modify the
statistical tensors to take feeding due to multiple excitation into
account.\\

\begin{equation}
H_k(I, I_n) =
{\sqrt{(2 I + 1) (2 I_n + 1)} \over \gamma(I)}
\sum_\lambda (-1)^{I + I_n + \lambda + k}
|\delta_\lambda|^2
(1 + c(\lambda))
\begin{Bmatrix}
I & I & k\\
I_n & I_n & \lambda
\end{Bmatrix}
\end{equation}

\noindent where $I_n$ is a level directly feeding level $I$ and where
$c(\lambda)$ are the internal conversion coefficients. Note that the
$\delta$ and $(1 + c(\lambda)$ terms are not included in this function, but
in {\em SEQ} (see section \ref{sect:seq}) which calls this function.\\

\noindent The term $\gamma(I)$ is the emission probability:\\

\begin{equation}
\gamma(I) =
\sum_{\lambda n} | \delta_\lambda(I_1 \rightarrow I_n)|^2
\end{equation}


\noindent It uses {\em WSIXJ} (see section \ref{sect:wsixj}) to evaluate
the Wigner 6-j coefficient.\\
