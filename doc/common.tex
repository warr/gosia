\chapter{Common blocks}

\section{A50}

This common block contains the required accuracy for integration. It is used
in {\em INTG} and {\em RANGEL}.

\begin{itemize}
\item REAL*8 ACC50 - required accuracy for integration.
\end{itemize}

\section{ADBXI}

This common block contains the adiabatic exponential. It is used in {\em
AMPDER}, {\em GOSIA}, {\em LAISUM}, {\em NEWLV} and {\em STING}.

\begin{itemize}
\item REAL*8 EXPO(500) - adiabatic exponential for each matrix element.
\end{itemize}

\section{ADX}

This common block contains the adiabatic parameters for each of the 365
possible different values of $\omega$. It is used in the functions {\em
EXPON} and {\em SETIN}. This is $\epsilon \sinh(\omega) + \omega$.\\

\begin{itemize}
\item REAL*8 ADB(365) - adiabatic parameter for each $\omega$.
\end{itemize}

\section{ALLC}

This common block is used in {\em ALLOC}, {\em LAISUM} and {\em SNAKE}.

\begin{itemize}
\item INTEGER*4 LOCQ(8,7) - The position of the start of the collision
coefficients for each ($\lambda$,$\mu$) relative to the start of the
collision coeficients, which is in the ZETA array starting at LP7 (45100).
\end{itemize}

\section{APRCAT}

This common block is used in {\em APRAM}, {\em CODE7}, {\em GOSIA}, {\em
LOAD}, {\em LSLOOP}, {\em NEWCAT}, {\em PODZIEL} and {\em POMNOZ}.

\begin{itemize}
\item REAL*8 QAPR(500,2,7)
\item INTEGER*4 IAPR(500,2) - indices of intitial and final levels for
matrix element.
\item INTEGER*4 ISEX(75)
\end{itemize}

\section{APRX}

This common block is used in {\em APRAM}, {\em GOSIA}, {\em PODZIEL} and
{\em POMNOZ}.

\begin{itemize}
\item INTEGER*4 LERF - error flag from expansion in {\em POMNOZ}.
\item INTEGER*4 IDIVE(50,2) - number of subdivisions for L=1 and L=2 for
each experiment for approximate calculation.
\end{itemize}

\section{AZ}

This common block contains the reduced matrix elements. It is used in {\em
AMPDER}, {\em APRAM}, {\em DOUBLE}, {\em FTBM}, {\em GOSIA}, {\em HALF},
{\em INTG}, {\em LAIAMP}, {\em LAISUM}, {\em POMNOZ}, {\em RESET}, {\em
SIXEL}, {\em STING} and {\em TENB}.

\begin{itemize}
\item COMPLEX*16 ARM(600,7) - excitation amplitudes of substates. It is used
in the Adams-Moulton predictor-corrector method. ARM(I,1) = f(n-3), ARM(I,2)
= f(n-2), ARM(I,3) = f(n-1), ARM(I,4) = f(n). ARM(I,5) is the corrector,
ARM(I,7) is the predictor. This is in {\em intg.f}. In {\em sting.f}
ARM(I,6) is also used.
\end{itemize}

\section{BREC}

This common block has the recoil $\beta$. It is used in {\em ANGULA}, {\em
CEGRY}, {\em CMLAB}, {\em GKVAC} and {\em GOSIA}.

\begin{itemize}
\item REAL*8 BETAR(50) - recoil beta for each experiment.
\end{itemize}

\section{BRNCH}

This common block has branching ratios. It is used in {\em ADHOC}, {\em
BRANR} and {\em GOSIA}.

\begin{itemize}
\item REAL*8 BRAT(50,2) - branching ratio and its error.
\item INTEGER*4 IBRC(2,50) - index branching ratios.
\item INTEGER*4 NBRA - number of branching ratios.
\end{itemize}

\section{CATLF}

This common block is used in {\em ANGULA}, {\em DECAY} and {\em SEQ}.

\begin{itemize}
\item FP(4,500,3) - F coefficient multiplied by $\delta_lambda^2$.
\item GKP(4,500,2) - G$_k$ multiplied by $\delta_lambda^2$.
\item KLEC(75) - number of decays for each level
\end{itemize}

\section{CAUX}

This common block is used in {\em AMPDER}, {\em INTG}, {\em LAIAMP}, {\em
LAISUM}, {\em NEWLV} and {\em STING}.

\begin{itemize}
\item INTEGER*4 NPT - index in ADB array (this is $\omega$ / 0.03).
\item INTEGER*4 NDIV - number of divisions for integration
\item INTEGER*4 KDIV - index of division
\item INTEGER*4 LAMR(8) - flag = 1 if we should calculate for this
multipolarity.
\item INTEGER*4 ISG - phase
\item REAL*8 D2W - step in $\omega$ = 0.03
\item INTEGER*4 NSW - number of sigma values?
\item INTEGER*4 ISG1 - index of sigma
\end{itemize}

\section{CAUX0}

This common block is used in {\em CMLAB}, {\em GOSIA} and {\em LOAD}.

\begin{itemize}
\item REAL*8 EMMA(75) - controls the number of magnetic substates to be
included in the full coulomb excitation calculation. This is {\em M$_C$} in
the documentation of record type ``EXPT''.
\item INTEGER*4 NCM - Assume final state of nucleus has this spin (default =
2.0). See OP,CONT switch ``NCM,''.
\end{itemize}

\section{CB}

This common block is used in {\em DJMM} and {\em GOSIA}

\begin{itemize}
\item REAL*8 B(20)
\end{itemize}

\section{CCC}

This common block contains the conversion coefficients. It is used in the
functions {\em ADHOC}, {\em ANGULA}, {\em CEGRY}, {\em CONV}, {\em FTBM}
and {\em GOSIA}, {\em }

\begin{itemize}
\item REAL*8 EG(50) - energies for which conversion coefficients are
provided.
\item REAL*8 CC(50, 5) - conversion coefficients for each energy and
multipolarity.
\item REAL*8 AGELI(50,200,2) - angles of Ge detectors.
\item REAL*8 Q(3,200,8) - solid angle attenuation coefficients.
\item INTEGER*4 NICC - number of conversion coefficients.
\item INTEGER*4 NANG(200) - number of $\gamma$-ray detectors.
\end{itemize}

\section{CCCDS}

This common block is used in {\em ADHOC}, {\em CEGRY}, {\em GOSIA} and {\em
READY}.

\begin{itemize}
\item INTEGER*4 NDST(50) - number of data sets for each experiment. Usually
equal to NANG, unless detector clusters were defined in OP,RAW
\end{itemize}

\section{CCOUP}

This common block contains the coupling coefficients $\zeta$ and also the
collision functions. Note that although there is only one array, different
parts of the array are used for different things! This common block is used
in {\em AMPDER}, {\em ANGULA}, {\em DECAY}, {\em FTBM}, {\em GOSIA}, {\em
KLOPOT}, {\em LAIAMP}, {\em LAISUM}, {\em LOAD}, {\em LSLOOP}, {\em SNAKE},
{\em STING}, {\em TAPMA} and {\em TENS}.

\begin{itemize}
\item ZETA(50000) - a variety of coefficients including $\zeta$, collision
coefficients etc.
\item LZETA(8) - points to the start of the $\zeta$ coefficients for each
multipolarity. The index is E1{\ldots}6, M1, M2 in that order.
\end{itemize}

\section{CEXC}

This common block is used in {\em ADHOC}, {\em APRAM}, {\em FTBM}, {\em
GOSIA}, {\em KLOPOT}, {\em KONTUR}, {\em LIMITS}, {\em LOAD}, {\em MINI},
{\em MIXUP}, {\em POMNOZ} and {\em PRELM}.

\begin{itemize}
\item INTEGER*4 MAGEXC
\item INTEGER*4 MEMAX - number of matrix elements
\item INTEGER*4 LMAXE - maximum multipolarity needed for calculation. It is
used by {\em LOAD} to limit the number of $\xi$ and $\psi$ parameters
calculated to those actually needed.
\item INTEGER*4 MEMX6 - number of matrix elements with E1{\ldots}6
multipolarity.
\item INTEGER*4 IVAR(500) - Indicates a limit or correlation is set.
\end{itemize}

\section{CEXC0}

This common block is used in {\em AMPDER}, {\em DOUBLE}, {\em FTBM}, {\em
HALF}, {\em INTG}, {\em LAIAMP}, {\em LAISUM}, {\em LOAD}, {\em LSLOOP},
{\em NEWLV}, {\em PATH}, {\em RESET} and {\em TENB}.

\begin{itemize}
\item INTEGER*4 NSTART(76) - NSTART(I) gives the index in the CAT and ARM
arrays for the first substate associated with the level I.
\item INTEGER*4 NSTOP(75) - NSTART(I) gives the index in the CAT and ARM
arrays for the last substate associated with the level I.
\end{itemize}

\section{CEXC9}

This common block is used in {\em GOSIA} and {\em INTG}.

\begin{itemize}
\item INTEGER*4 INTERV(50)
\end{itemize}

\section{CHI1T}

This common block is used in {\em FTBM} and {\em MINI}.

\begin{itemize}
\item REAL*8 CHIS11
\end{itemize}

\section{CINIT}

This common block contains the normalisation parameters. It is used in {\em
CEGRY} and {\em GOSIA}.

\begin{itemize}
\item REAL*8 CNOR(32,75) - normalisation for each $\gamma$ detector and each
level.
\item INTEGER*4 INNR - independent normalisation switch. See OP,CONT switch
``INR,''.
\end{itemize}

\section{CLCOM}

This common block is used in {\em AMPDER}, {\em BRANR}, {\em GOSIA}, {\em
LAIAMP}, {\em LEADF}, {\em LOAD}, {\em MEM}, {\em NEWCAT}, {\em NEWLV}, {\em
PRELM}, {\em PTICC}, {\em QRANGE}, {\em RANGEL}, {\em SEQ} and {\em STING}.

\begin{itemize}
\item INTEGER*4 LAMDA(8) - list of multipolarities to calculate
\item INTEGER*4 LEAD(2,500) - pair of levels involved in each matrix element
\item INTEGER*4 LDNUM(8,75) - number of matrix elements with each multipolarity populating each level
\item INTEGER*4 LAMMAX - number of multipolarities to calculate
\item INTEGER*4 MULTI - number of matrix elements for a given multipolarity
\end{itemize}

\section{CLCOM0}

This common block is used in {\em GOSIA}, {\em INTG} and {\em LSLOOP}.

\begin{itemize}
\item INTEGER*4 IFAC(75) - spin/parity phase factor for level relative to
ground state. Is +1 for even spin difference and no parity difference or odd
spin difference and different parity and -1 otherwise.
\end{itemize}

\section{CLCOM8}

This common block is used in {\em AMPDER}, {\em DOUBLE}, {\em FTBM}, {\em
GOSIA}, {\em HALF}, {\em INTG}, {\em LAIAMP}, {\em LAISUM}, {\em LOAD}, {\em
LSLOOP}, {\em PATH}, {\em RESET}, {\em STING} and {\em TENB}.

\begin{itemize}
\item REAL*8 CAT(600,3) - this array has one entry for each substate.
CAT(I,1) is the level number associated with the substate, CAT(I,2) is the
spin of that level and CAT(I,3) is the $m$ quantum number of the substate.
\item INTEGER*4 ISMAX - number of substates in CAT which are used.
\end{itemize}

\section{CLCOM9}

This common block is used in {\em CMLAB} and {\em GOSIA}.

\begin{itemize}
\item REAL*8 ERR - error flag, set in {\em CMLAB} if the maximum scattering
angle or excitation energy is exceeded.
\end{itemize}

\section{CLM}

This common block is used in {\em FTBM}, {\em GOSIA} and {\em LOAD}.

\begin{itemize}
\item INTEGER*4 LMAX
\end{itemize}

\section{CLUST}

This common block is used in {\em CEGRY} and {\em GOSIA}.

\begin{itemize}
\item INTEGER*4 ICLUST(50,200)
\item INTEGER*4 LASTCL(50,20)
\item REAL*8 SUMCL(20,500)
\end{itemize}

\section{COEX}

\begin{itemize}
\item 
\end{itemize}

This common block contains information about levels. It is used in {\em
ADHOC}, {\em BRANR}, {\em CEGRY}, {\em CMLAB}, {\em FTBM}, {\em GKVAC}, {\em
GOSIA}, {\em INTG}, {\em LAISUM}, {\em LOAD}, {\em LSLOOP}, {\em PRELM},
{\em PTICC}, {\em SEQ} and {\em TENB}.

\begin{itemize}
\item REAL*8 EN(75) - the energy of each level.
\item REAL*8 SPIN(75) - the spin of each level.
\item REAL*8 ACCUR - accuracy required. This is set via OP,CONT switch
``ACC,''. ACCUR contains 10$^x$ where $x$ is the value supplied by the user.
Default is 0.00001.
\item REAL*8 DIPOL - E1 polarisation parameter. This is 0.001 times the
value given by the user with the OP,CONT ``DIP,'' switch. Default is 0.005.
\item REAL*8 ZPOL - dipole term
\item REAL*8 ACCA - accuracy.
\item REAL*8 ISO - isotropic flag
\end{itemize}

\section{COEX2}

This common block is used in {\em AMPDER}, {\em CEGRY}, {\em CMLAB}, {\em
DECAY}, {\em DOUBLE}, {\em FTBM}, {\em GOSIA}, {\em HALF}, {\em INTG}, {\em
LOAD}, {\em NEWCAT}, {\em PATH}, {\em PRELM}, {\em RESET}, {\em SEQ}, {\em
TENB} and {\em TENS}.

\begin{itemize}
\item INTEGER*4 NMAX - number of levels used
\item INTEGER*4 NDIM
\item INTEGER*4 NMAX1 - number of levels with decays
\end{itemize}

\section{COMME}

This common block holds the matrix elements and their limits and
correlations. It is used in {\em AMPDER}, {\em APRAM}, {\em BRANR}, {\em
CHMEM}, {\em DECAY}, {\em FTBM}, {\em GOSIA}, {\em KLOPOT}, {\em KONTUR},
{\em LAIAMP}, {\em LAISUM}, {\em LIMITS}, {\em LSLOOP}, {\em MINI}, {\em
MIXR}, {\em MIXUP}, {\em PRELM} and {\em STING}.

\begin{itemize}
\item REAL*8 ELM(500) - matrix elements.
\item REAL*8 ELMU(500) - upper limit on matrix elements.
\item REAL*8 ELML(500) - lower limit on matrix elements.
\item REAL*8 SA(500) - ratio of matrix elements for correlated elements.
\end{itemize}

\section{CX}

This common block has information about each experiment. The Z and A of
projectile and target, the bombarding energy and the $\theta$ angle of the
particle detector. It is used in the functions {\em ADHOC}, {\em CEGRY},
{\em CMLAB}, {\em COORD}, {\em FTBM}, {\em GKVAC}, {\em GOSIA}, {\em
KLOPOT}, {\em LOAD} and {\em READY}.

\begin{itemize}
\item INTEGER*4 NEXPT - number of experiments (1{\ldots}50). This is {\em
NEXPT} in the documentation of record type ``EXPT''.
\item INTEGER*4 IZ - Z of investigated nucleus. This is {\em Z$_1$} in the
documentation of record type ``EXPT''.
\item REAL*8 XA - A of investigated nucleus. This is {\em A$_1$} in the
documentation of record type ``EXPT''.
\item INTEGER*4 IZ1(50) - Z of non-investigated nucleus for each experiment.
This is {\em Z$_n$} in the documentation of record type ``EXPT''.
\item REAL*8 XA1(50) - A of non-investigated nucleus for each experiment.
This is {\em A$_n$} in the documentation of record type ``EXPT''.
\item REAL*8 EP(50) - bombarding energy for each experiment. This is {\em
E$_P$} in the documentation of record type ``EXPT''.
\item REAL*8 TLBDG(50) - $\theta$ angle of particle detector for each
experiment in the laboratory frame. This is {\em $\theta_{LAB}$} in the
documentation of record type ``EXPT''.
\item REAL*8 VINF(50) - speed of projectile at infinity for each experiment.
\end{itemize}

\section{CXI}

This common block holds the $\xi$ coupling coefficients. It is used in {\em
EXPON}, {\em FXIS1}, {\em FXIS2}, {\em GOSIA}, {\em LAIAMP}, {\em LOAD} and
{\em NEWCAT}.

\begin{itemize}
\item REAL*8 XI(500) - $\xi$ coupling coefficients.
\end{itemize}

\section{DFTB}

This common block is used in {\em GOSIA}, {\em KONTUR} and {\em MINI}.

\begin{itemize}
\item REAL*8 DEVD(500)
\item REAL*8 DEVU(500)
\end{itemize}

\section{DIMX}

This common block has the parameters provided by OP,GDET. It is used in {\em
GCF}, {\em ADHOC}, {\em CEGRY} and {\em GOSIA}.

\begin{itemize}
\item REAL*8 A - radius of inactive p-core (the inner radius of a crystal).
\item REAL*8 R - radius of the active n-core (the outer radius of a crstal).
\item REAL*8 XL - length of a crystal.
\item REAL*8 D - distance from a target to the face of a crystal.
\item REAL*8 ODL(200) - results of OP,GDET calculation.
\end{itemize}

N.B. Sometimes A, R, XL, D are referred to as DIX(4).

\section{DUMM}

This common block is used in {\em GOSIA} and {\em MINI}.

\begin{itemize}
\item REAL*8 GRAD(500) - partial derivatives of $\xi^2$.
\item REAL*8 HLMLM(500) - old value of matrix element or $\xi^2$ (it is used
for both).
\item REAL*8 ELMH(500)
\end{itemize}

\section{EFCAL}

This common block holds the parameters for describing absorbers. It is used
in {\em EFFIX} and {\em GOSIA}.

\begin{itemize}
\item REAL*8 ABC(8,10) - absorption coefficients.
\item REAL*8 AKAVKA(8,200) - efficiency curve parameters.
\item REAL*8 THICK(200,7) - thickness of each absorber type.
\end{itemize}

\section{ERCAL}

This common block is used in {\em GOSIA} and {\em MINI}.

\begin{itemize}
\item INTEGER*4 JENTR
\item INTEGER*4 ICS - flag: read internal correction factors (see OP,CONT
switch ``CRF,'').
\end{itemize}

\section{ERRAN}

THis common block is used in {\em GOSIA} and {\em MINI}.

\begin{itemize}
\item INTEGER*4 KFERR - error flag for minimisation.
\end{itemize}

\section{FAKUL}

This common block is used in {\em FAKP}, {\em GOSIA}, {\em PRIM}, {\em
WSIXJ} and {\em WTHREJ}.

\begin{itemize}
\item INTEGER*4 IP(26) - table of prime numbers
\item INTEGER*4 IPI(26) - number of time a number is divisible by each prime
in IP.
\item INTEGER*4 KF(101,26) - sum of factors of primes.
\item REAL*8 PILOG(26) - table of natural logs of primes.
\end{itemize}

\section{FIT}

This common block holds the parameters for OP,MINI. It is used in {\em
GOSIA} and {\em MINI}. These parameters are documented in the input
parameters to OP,MINI. Note, however, that DLOCK is called DLOCKS there.

\begin{itemize}
\item INTEGER*4 LOCKF - flag: if LOCKF = 0, terminate minimisation of convergence limit
CONV reached. If LOCKF = 1 fix the NLOCK matrix elements having the most
significant S derivatives.
\item INTEGER*4 NLOCK - number of matrix elements having the largest
derivatives of S to be locked if LOCKF = 1 and the convergence limit CONV is
satisfied.
\item INTEGER*4 IFBFL - flag: IFBFL = 0 means derivatives calculated using
only the forward difference method. IFBFL = 1 means use forward-backward
method.
\item INTEGER*4 LOCKS - flag: LOCKS = 1 means fix at first stage of
minimisation all elements with absolute value of the partial derivative less
than DLOCKS.
\item REAL*8 DLOCK - the limit of the derivative.
\end{itemize}

\section{FLA}

This common block is used in {\em INTG}, {\em NEWLV} and {\em STING}.

\begin{itemize}
\item INTEGER*4 IFLG - flag: determines whether {NEWLV} should calculate
exponentials (so we only perform the calculation once).
\end{itemize}

\section{GGG}

This common block holds the parameters for the vacuum polarisation. It is
used in {\em GKK}, {\em GKVAC} and {\em GOSIA}.

\begin{itemize}
\item REAL*8 AVJI - J$_1$
\item REAL*8 GAMMA - $\Gamma$
\item REAL*8 XLAMB - $\Lambda^*$
\item REAL*8 TIMEC - $\tau_c$
\item REAL*8 GFAC - g
\item REAL*8 FIEL - K
\item REAL*8 POWER - x
\end{itemize}

\section{GVAC}

This common block is used in {\em GKK} and {\em GKVAC}.

\begin{itemize}
\item REAL*8 GKI(3) - $G_k$ for a single level. Perhaps we should just pass
this array as a parameter.
\item REAL*8 SUM(3) - Sum over square of 6-j symbol used to calculate
$<\alpha_k>$.
\end{itemize}

N.B. It doesn't look as though {\em SUM} is used anywhere except {\em GKK}.

\section{HHH}

This common block is used in {\em GOSIA}, {\em KONTUR} and {\em PRELM}.

\begin{itemize}
\item REAL*8 HLM(500) - the matrix elements. What is the difference between
ELM, HLM and HLMLM?
\end{itemize}

\section{HIPER}

The common block {\em HIPER} has a table of 365 values for the hyperbolic
functions $\sinh\omega$ and $\cosh\omega$. It is used in the functions {\em
FHIP}, {\em SETIN} and {\em SNAKE}.

\begin{itemize}
\item REAL*8 SH(365) - $\sinh\omega$
\item REAL*8 CH(365) - $\cosh\omega$
\end{itemize}

\section{IDENT}

This common block is used in {\em DJMM} and {\em GOSIA}. It has something to
do with the $\beta$ of the rotation matrices. It is initialised to -983872.
in {\em GOSIA}, but in {\em DJMM} it is compared with $\pi$.

\begin{itemize}
\item REAL*8 BEQ - identifier used to match $\beta$ so we can tell if we
have already calculuated this value of beta in {\em DJMM}.
\end{itemize}

\section{IGRAD}

This common block is used in {\em CEGRY} and {FTBM}.

\begin{itemize}
\item INTEGER*4 IGRD
\end{itemize}

\section{ILEWY}

This common block is used in {\em KONTUR} and {\em MINI}.

\begin{itemize}
\item INTEGER*4 NWR
\end{itemize}

\section{INHI}

This common block is used in {\em GOSIA} and {\em POMNOZ}.

\begin{itemize}
\item INTEGER*4 INHB
\end{itemize}


\section{LCDL}

This common block is used in {\em ANGULA} and {\em DECAY}.

\begin{itemize}
\item REAL*8 DELLA(500,3) - products of matrix elements $e_^2$, $e_2^2$ and
$e_1 \times e_2$.
\end{itemize}

\section{LCZP}

This common block is used in {\em CEGRY}, {\em FTBM} and {\em MINI}.

\begin{itemize}
\item REAL*8 EMH
\item INTEGER*4 INM
\item INTEGER*4 LFL1
\item INTEGER*4 LFL2
\item INTEGER*4 LFL
\end{itemize}

\section{LEV}

This common block is used in {\em ADHOC}, {\em ANGULA}, {\em BRANR}, {\em
CEGRY}, {\em DECAY}, {\em FTBM}, {\em GKVAC}, {\em GOSIA}, {\em MIXR}, {\em
PTICC}, {\em READY} and {\em SEQ}.

\begin{itemize}
\item REAL*8 TAU(75) - lifetime in picoseconds, but also used for energies
in {\em SEQ}.
\item INTEGER*4 KSEQ(500,4) - For each decay: level1, level2, matrix element
and multipolarity + 10 in {\em SEQ} but something else otherwise!
\end{itemize}

\section{LIFE}

This common block holds the number of lifetimes. It is used in {\em ADHOC},
{\em CEGRY} and {\em FTBM}.

\begin{itemize}
\item INTEGER*4 NLIFT - number of lifetimes
\end{itemize}

\section{LIFE1}

This common block is used in {\em ADHOC}, {\em DECAY} and {\em GOSIA}.

\begin{itemize}
\item INTEGER*4 LIFCT(50) - index of level for lifetimes.
\item REAL*8 TIMEL(2,50) - lifetimes and their errors.
\end{itemize}

\section{LOGY}

This common block is used in {\em FTBM}, {\em GOSIA}, {\em KONTUR}, {\em
MINI} and {\em MIXR}.

\begin{itemize}
\item INTEGER*4 LNY - flag: use logs to calculate $\chi^2$.
\item INTEGER*4 INTR
\item INTEGER*4 IPS1 - flag: terminate after calculating and writing
correction factors. (See OP,CONT switch ``CCF,'').
\end{itemize}

\section{KIN}

This common block contains the kinematic parameters. It is used in the
functions {\em ANGULA}, {\em CEGRY}, {\em CMLAB}, {\em COORD}, {\em DECAY},
{\em FTBM}, {\em GKVAC}, {\em GOSIA}, {\em LAIAMP}, {\em SETIN}, {\em
SIXEL}, {\em SNAKE}, {\em TENB} and {\em YLM}.

\begin{itemize}
\item REAL*8 EPS(50) - orbit excentricity, $\epsilon$, for each experiment.
\item REAL*8 EROOT(50) - $\sqrt{\epsilon^2 -1}$ for each experiment.
\item REAL*8 FIEX(50,2) - $\phi_{min}$ and $\phi_{max}$ for the particle
detector in radians. Note that the user enters these values in degrees, but
the program converts. These correspond to $\phi_1$ and $\phi_2$ in the
documentation of record type ``EXPT''.
\item INTEGER*4 IEXP - the current experiment number 1{\ldots}50.
\item INTEGER*4 IAXS(50) - The axial symmetry flag for each experiment. This
is {\em IAX} in the documentation for record type ``EXPT''.
\end{itemize}

\section{MAP}

This common block is used in {\em GOSIA} and {\em NEWCAT}.

\begin{itemize}
\item REAL*8 PARX(50,12,5)
\item REAL*8 PARXM(50,4,10,6)
\item REAL*8 XIR(6,50)
\end{itemize}

\section{ME2D}

This common block holds information about the known E2 matrix elements
provided by the user. It is used in {\em ADHOC}, {\em CHMEM} and {\em
GOSIA}.

\begin{itemize}
\item REAL*8 EAMX(100,2) - known E2 matrix elements and their error.
\item INTEGER*4 NAMX - number of known E2 matrix elements.
\item INTEGER*4 IAMX(100) - index of matrix element, which this corresponds
to.
\item INTEGER*4 IAMY(100,2) - initial and final state for each known matrix
element.
\end{itemize}

The matrix elements are in ELM(500). IAMX(i) holds the index into ELM of the
i$^{th}$ known matrix element. So ELM(IAMX(i)) is the value of that matrix
element, which should correspond to EAMX(i,1) $\pm$ EAMX(i,2).

\section{MGN}

This common block has a number of different limits and offsets. Most are
hardcoded constants within the program. It is used in the functions {\em
ALLOC}, {\em CEGRY}, {\em FHIP}, {\em FTBM}, {\em GOSIA}, {\em KLOPOT}, {\em
LAISUM}, {\em LOAD}, {\em LSLOOP}, {\em MINI}, {\em PODZIEL}, {\em READY},
{\em SEQ}, {\em SETIN} and {\em SNAKE}.

\begin{itemize}
\item INTEGER*4 LP1 - maximum number of experiments = 50.
\item INTEGER*4 LP2 - maximum number of matrix elements = 500.
\item INTEGER*4 LP3 - maximum number of levels = 75.
\item INTEGER*4 LP4 - maximum number of yields = 1500.
\item INTEGER*4 LP6 - maximum number of $\gamma$ detectors = 32.
\item INTEGER*4 LP7 - offset of collision functions in ZETA array (45100).
\item INTEGER*4 LP8 - ? = 104.
\item INTEGER*4 LP9 - length of ZETA array - 2100 = 47900.
\item INTEGER*4 LP10 - maximum number of reduced matrix elements = 600.
\item INTEGER*4 LP11 - LP8 - 1 = 103.
\item INTEGER*4 LP12 - Maximum number of steps of $\omega$ = 365.
\item INTEGER*4 LP13 = LP9 + 1 = 47901.
\item INTEGER*4 LP14 = Maximum space for collision functions = 4900.
\end{itemize}

\section{MINNI}

This common block is used in {\em CEGRY} and {\em GOSIA}.

\begin{itemize}
\item INTEGER*4 IMIN
\item INTEGER*4 LNORM(50) - Fixes the normalisation factor of a given
experiment to another experiment. i.e. if LNORM(3) has the value 2, it means
that the normalisation for experiment 3 has a fixed ratio to that of
experiment 2. This is {\em LN} in the documentation for record type ``EXPT''.
\end{itemize}

\section{MIXD}

This common block holds the multipole mixing ratio data supplied by the
user. It is used in {\em ADHOC} and {\em MIXR}.

\begin{itemize}
\item REAL*8 DMIXE(20,2) - mixing ratio and its error.
\item REAL*8 DMIX(20) - 0.8326 $\times$ $\Delta$E.
\item INTEGER*4 IMIX(20)
\item INTEGER*4 NDL - the number of mixing ratios.
\end{itemize}

\section{ODCH}

This common block is used in {\em CEGRY} and {\em sixel}.

\begin{itemize}
\item REAL*8 DEV(500)
\end{itemize}

\section{PCOM}

This common block contains the $\psi$ values. It is used in {\em LOAD} and
{\em LSLOOP}.

\begin{itemize}
\item REAL*8 PSI(500) - $\psi$.
\end{itemize}

\section{PINT}

This common block is used in {\em AMPDER}, {\em LAISUM}, {\em NEWLV} and
{\em STING}.

\begin{itemize}
\item INTEGER*4 ISSTAR(76) - first substate for a given level.
\item INTEGER*4 ISSTO(75) - last substate for a given level.
\item INTEGER*4 MSTORE(2,75) - MSTORE(1,75) is index of final level and
MSTORE(2,75) is index of matrix element in ELM.
\end{itemize}

\section{PRT}

This common block contains an array of flags which control the output
generated by gosia. It is used in {\em ADHOC}, {\em BRANR}, {\em CEGRY},
{\em CMLAB}, {\em FTBM}, {\em GOSIA} and {\em MINI}.

\begin{itemize}
\item INTEGER*4 IPRM(20) - various flags which control printing. These are
documented in the OP,CONT switch ``PRT,''
\end{itemize}

\section{PSPIN}

This common block is used in {\em LAISUM} and {\em LOAD}.

\begin{itemize}
\item INTEGER*4 ISHA
\end{itemize}

\section{PTH}

This common block is used in {\em APRAM}, {\em CODE7}, {\em FTBM}, {\em
GOSIA}, {\em INTG}, {\em LOAD}, {\em LSLOOP}, {\em NEWCAT}, {\em PATH} and
{\em POMNOZ}.

\begin{itemize}
\item INTEGER*4 IPATH(75) - index of substate in level with same m as
normalisation substate.
\item INTEGER*4 MAGA(75) - Controls the number of magnetic substates to be
included in the approximate coulomb excitation calculation. This is {\em
M$_A$} in the documentation of record type ``EXPT''.
\end{itemize}

\section{RNG}

This common block is used in {\em ALLOC}, {\em INTG}, {\em QRANGE}, {\em
RANGEL} and {\em STING}.

\begin{itemize}
\item INTEGER*4 IRA(8) - range of omega needed for each multipolarity.
\item INTEGER*4 MAXLA
\end{itemize}

\section{SECK}

This common block is used in {\em CMLAB}, {\em COORD} and {\em GOSIA}.

\begin{itemize}
\item INTEGER*4 ISKIN(50) - kinematic flag: if A$_{projectile}$ $>$
A$_{target}$, ISKIN specifies which of the two possible solutions to choose.
This is {\em IKIN} in the documentation for record type ``EXPT''
\end{itemize}

\section{SEL}

This common block is used in {\em CEGRY}, {\em GOSIA}, {\em KLOPOT}, {\em
MINI} and {\em SIXEL}.

\begin{itemize}
\item INTEGER*4 KVAR(500)
\end{itemize}

\section{SKP}

This common block is used in {\em CEGRY}, {\em FTBM} and {\em GOSIA}.

\begin{itemize}
\item INTEGER*4 JSKIP(50) - Experiments to skip during minimisation (see
OP,CONT switch ``SKP,'').
\end{itemize}

\section{TCM}

This common block is used in {\em CEGRY}, {\em CMLAB}, {\em GOSIA} and {\em
TENS}.

\begin{itemize}
\item REAL*8 TETACM(50) - $\theta$ of particle detector in centre of mass
frame.
\item REAL*8 TREP(50) - $\theta$ of recoiling nucleus.
\item REAL*8 DSIGS(50) - d$\sigma$.
\end{itemize}

\section{THTAR}

This common block is used in {\em ANGULA}, {\em GKVAC} and {\em GOSIA}.

\begin{itemize}
\item INTEGER*4 ITTE(50) - thick target experiment flag.
\end{itemize}

\section{TRA}

This common block is used in {\em ADHOC}, {\em ANGULA}, {\em BRANR}, {\em
CEGRY}, {\em DECAY}, {\em GOSIA} and {\em SEQ}.

\begin{itemize}
\item REAL*8 DELTA(500,3) - $\delta^2_{\lambda_E}$, $\delta^2_{\lambda_M}$,
$\delta_{\lambda_E}\times\delta_{\lambda_M}$.
\item REAL*8 ENDEC(500) - energy difference between levels for each matrix
element.
\item INTEGER*4 ITMA(50,200) - identity of each detector for each
experiment. This is used as an index for the array ODL and as a parameter
for the function {\em EFFIX}.
\end{itemize}

\section{TRB}

This common block is used in {\em ADHOC}, {\em CEGRY}, {\em GOSIA} and {\em
SIXEL}.

\begin{itemize}
\item INTEGER*4 ITS - flag: create tape 18 file.
\end{itemize}

\section{UWAGA}

This common block is used in {\em FTBM} and {\em MINI}.

\begin{itemize}
\item INTEGER*4 ITAK2
\end{itemize}

\section{VAC}

This common block holds the vacuum polaristation parameters. It is used in
{\em CEGRY}, {\em DECAY}, {\em GKK}, {\em GKVAC}, {\em GOSIA} and {\em
XSTATIC}.

\begin{itemize}
\item REAL*8 VACDP(3,75) - G$_k$ for each level.
\item REAL*8 QCEN - centre of gaussian distribution.
\item REAL*8 DQ - width of gaussian distribution.
\item REAL*8 XNOR - normalisation factor.
\item REAL*8 AKS(6,75) - <\alpha_k> for each level.
\item INTEGER*4 IBYP - flag to indicate whether to calculate <\alpha_k> or
to use previously calculated values.
\end{itemize}

\section{VLIN}

This common block is used in {\em COORD}, {\em GOSIA}, {\em KONTUR} and {\em
TAMPA}. It has something to do with the sensitivity maps.

\begin{itemize}
\item REAL*8 XV(51) - energy meshpoints at which the exact calculation is
performed.
\item REAL*8 YV(51) - projectile scattering angle meshpoints in the lab
frame, at which the exact calculation is performed.
\item REAL*8 ZV(20) - energy meshpoints.
\item REAL*8 DSG(20) - differential $\gamma$-ray yield at meshpoints
\item REAL*8 DSE(20) - Rutherford cross section at given energy integrated
over angles.
\item REAL*8 DS - integrated rutherford cross section.
\end{itemize}

\section{WARN}

This common block is used in {\em CEGRY} and {\em GOSIA}.

\begin{itemize}
\item REAL*8 SGW - this is a switch, which is 3 by default and can be
changed using the OP,CONT flag WRN.
\item REAL*8 SUBCH1 - $\chi^2$ subtotal.
\item REAL*8 SUBCH2 - $\chi^2$ subtotal.
\item INTEGER*4 IWF - warning flag.
\end{itemize}

\section{XRA}

This common block holds the seed for the random number generator used to mix
up the wavefunctions. It is used in {\em GOSIA} and {\em MIXUP}.

\begin{itemize}
\item REAL*8 SE - random number seed.
\end{itemize}

\section{YEXPT}

This common block holds the experimental yields. It is used in {\em ADHOC},
{\em CEGRY}, {\em FTBM}, {\em GOSIA}, {\em KLOPOT}, {\em MINI}, {\em READY}
and {\em SZEREG}.

\begin{itemize}
\item REAL*8 YEXP(32,1500) - experimental yields for each detector.
\item INTEGER*4 IY(1500,32) - index for yields
\item REAL*8 CORF(1500,32) - internal correction factors.
\item REAL*8 DYEX(32,1500) - error on experimental yields for each detector.
\item INTEGER*4 NYLDE(50,32) - number of yields for each detector and each
experiment.
\item REAL*8 UPL(32,50) - upper limits for all $\gamma$ detectors in each
experiment.
\item REAL*8 YNRM(32,50) - relative normalisation factors of $\gamma$
detectors in each experiment.
\item INTEGER*4 IDRN - index in yields to normalising transition
\item INTEGER*4 ILE(32) - index to start of yield information for each
detector. The values in ILE are used to index into IY, CORF, DYEX and YEXP.
\end{itemize}

\section{YTEOR}

This common block holds the calculated yields. It is used in {\em ADHOC},
{\em CEGRY}, {\em GOSIA} and {\em TAPMA}.

\begin{itemize}
\item REAL*8 YGN(500) - $\gamma$ yield without correction to angular
distribution for finite recoil distance.
\item REAL*8 YGP(500) - $\gamma$ yield with correction to angular
distribution for finite recoil distance. Only calculated if IFMO is 1.
\item INTEGER*4 IFMO - include correction to angular distribution for finite
recoil distance. This is {\em IFLAG} in the documentation to OP,YIEL.
\end{itemize}
