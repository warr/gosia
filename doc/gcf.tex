\section{GCF}
\label{sect:gcf}

\noindent This function calculates the detection probability, which is the
probability that the $\gamma$ ray is absorbed in the Ge but not in one of
the absorbers.\\

\noindent Following the method of Krane {\em Nucl. Instr. Meth} 98 (1972)
205-210, we introduce $\gamma$ energy-dependent attenuation factors
$Q_k(E_\gamma$) multiplying the final decay statistical tensors, defined as:

\begin{equation}
Q_k(E_\gamma) = {J_k(E_\gamma) \over J_0(E_\gamma)}
\end{equation}

\noindent where:\\

\begin{equation}
J_k = \int_0^{\alpha_{max}}
P_k(\cos\alpha)
K(\alpha)
(1 - e^{-\tau(E_\gamma)x(\alpha)})
\sin(\alpha)
d\alpha
\end{equation}

\noindent and $\alpha$ is the angle between the detector symmetry axis and
the $\gamma$-ray direction.\\

\noindent Note that this method is highly dependent on the detector geometry
which is assumed to be coaxial. Note that this method does not take the
closed front end of the Ge detector into account. The code uses the radius
of the core $a$, the outer radius of the Ge detector $r$, the length of the
detector $l$ (XL in the code) and the distance $d$. These are taken from the
array {\em DIX} in the common block {\em DIMX}.\\

\noindent We calculate the angle for the front edge of the detector, the
front edge of the core electrode (assuming it to protrude right to the front
face of the detector) and the back edges of the detector and the core
electrode. We then integrate across the angular range and along the path at
that angle, calculating the absorption probability in the absorbers and in
the Ge. We have to multiply with the Legendre polynomials summing from first
to ninth order. These polynomials are hardcoded into this function.\\

\begin{center}
\begin{tabular}{|ll|}
\hline
k & P$_k(\cos\theta)$\\
\hline
1 & 1\\
2 & $\cos\theta$\\
3 & 1.5$\cos^2\theta$ -0.5\\
4 & 2.5$\cos^3\theta$-1.5$\cos\theta$\\
5 & 4.375$\cos^4\theta$-3.75$\cos^2\theta$+0.375\\
6 & (63$\cos^5\theta$-70$\cos^3\theta$+15)/8\\
7 & (21$\cos^2\theta$(11$\cos^4\theta$-15$\cos^2\theta$+5)-5)/16\\
8 & (429$\cos^7\theta$-693$\cos^5\theta$+315$\cos^3\theta$-35$\cos\theta$)/16\\
9 & (6435$\cos^8\theta$-12012$\cos^6\theta$+6930$\cos^4\theta$-1260$\cos^2\theta$+35)/128\\
\hline
\end{tabular}
\end{center}

\noindent The integration is performed numerically using Simpson's rule over
100 intervals in each region of integration.\\

\noindent The fortran code in Gosia's {\em GCF} function is essentially the
same as the code with the same name in Krane's paper.\\
