\chapter{Ideas for improvements}

More comments and documentation!\\

Separate the different things in the \emph{ZETA} array.\\

Unravel some of the IF THEN statements in gosia.f, so that the logic is
easier to understand. At the moment we have several layers of IFs testing
the same variable, which could become IF {\ldots} ELSEIF {\ldots}
ELSIF{\ldots} ENDIF.\\

Try to decouple some of the variables in common blocks from the functions.
For example, \emph{YLM} need common block \emph{KIN} only for the axial
symmetry flag for the current experiment, which could be passed to it as a
formal parameter of the function.\\

Why do we need \emph{YLM} and \emph{YLM1}? Why don't we have a single function
to generate all these values and pass flags to it, to turn off the
calculation of those which we are not going to used.\\

In \emph{GKK}, the variable \emph{SUM} is in a common block, but it doesn't
seem to be used anywhere else, so why don't we just make it STATIC?\\

In lines 1909-1910 of gosia.f we have to successive tests on iosr, which
could be merged into an IF (iosr.eq.1) THEN {\ldots} ENDIF construction.\\

In lines 1680-1681 of gosia.f we have successive tests on op1.eq.`PIN,'
which could be merged into an IF (op1.eq.'PIN,') THEN {\ldots} ENDIF
construction.\\

There are some integer flags, which should be replaced with logicals. This
would prevent them accidently being given values $>$ 1.\\

The limit on NT in OP,INTG is 11 due to the dimensions of fiex1 and wpi in
gosia.f and Wpi in coord.f.\\

